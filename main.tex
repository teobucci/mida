\documentclass[10pt,a4paper,oneside,openright]{book}

% TODO
% among us
% uniformare Example e Observation bold/italic

%%%%%%%%%%%%%%%%%%%%%%%%%%%%%%%%%%%%%%%%%
% Template Dispense
% Autore: Teo Bucci
%%%%%%%%%%%%%%%%%%%%%%%%%%%%%%%%%%%%%%%%%

%----------------------------------------------------------------------------------------
%	FONTS AND LANGUAGE
%----------------------------------------------------------------------------------------

\usepackage[T1]{fontenc} % Use 8-bit encoding that has 256 glyphs
\usepackage[utf8]{inputenc} % Required for including letters with accents
\usepackage[english]{babel} % Italian language/hyphenation

%----------------------------------------------------------------------------------------
%	PACKAGES
%----------------------------------------------------------------------------------------

\usepackage{amsmath,amssymb,amsthm} % amssymb carica anche amsfonts

\usepackage{dsfont} % per funzione indicatrice
\newcommand{\Ind}{\mathds{1}}
%\usepackage{tikz-among-us}

\usepackage{booktabs}
\usepackage{tikz}
\usetikzlibrary{positioning}
\usetikzlibrary{shapes.misc}
\usetikzlibrary{intersections}
\usetikzlibrary{shapes.symbols}
\usepackage{mathdots}
\usepackage{cancel}
\usepackage{color}
\usepackage{siunitx}
\usepackage{array}
\usepackage{multirow}
\usepackage{makecell}
\usepackage{tabularx}
%\usepackage{booktabs}
\usepackage{caption}\captionsetup{belowskip=12pt,aboveskip=4pt}
\usepackage{subcaption}
\usepackage{placeins} % The placeins package gives the command \FloatBarrier, which will make sure any floats will be put in before this point
\usepackage{flafter}  % The flafter package ensures that floats don't appear until after they appear in the code.

\usepackage{import}
\usepackage{pdfpages}
\usepackage{transparent}
\usepackage{xcolor}
\usepackage{graphicx}
\graphicspath{ {./images/} }
\usepackage{float}
\usepackage[italian]{varioref}

\newcommand{\fg}[3][\relax]{%
  \begin{figure}[H]%[htp]%
    \centering
    \captionsetup{width=0.7\textwidth}
      \includegraphics[width = #2\textwidth]{#3}%
      \ifx\relax#1\else\caption{#1}\fi
      \label{#3}
  \end{figure}%
  \FloatBarrier%
}

%----------------------------------------------------------------------------------------
%	PARAGRAFI, INTERLINEA E MARGINE
%----------------------------------------------------------------------------------------

\usepackage[none]{hyphenat} % Per non far andare a capo le parole con il trattino

\emergencystretch 3em % Per evitare che il testo vada oltre i margini

% \parindent 0ex % Toglie intendamento paragrafi, e' incluso nel pacchetto \parskip
% \setlength{\parindent}{4em} % Variante di quello sopra

% \setlength{\parskip}{\baselineskip} % Cambia spazio tra paragrafi (posso mettere anche 1em) inclusa la table of contents, pertanto uso il comando che segue

\usepackage[skip=0.2\baselineskip+2pt]{parskip}

% \renewcommand{\baselinestretch}{1.5} % Cambia interlinea

%----------------------------------------------------------------------------------------
%	HEADERS AND FOOTERS
%----------------------------------------------------------------------------------------

\usepackage{fancyhdr}

\pagestyle{empty} % Il fancy serve a partire al primo capitolo
\fancyhead{} % Pulisci header
\fancyfoot{} % Pulisci footer

\fancyhead[RE]{\nouppercase{\leftmark}}
%\fancyhead[LO]{}
\fancyhead[LO]{\nouppercase{\rightmark}}
\fancyhead[LE,RO]{\thepage}

% Removes the header from odd empty pages at the end of chapters
\makeatletter
\renewcommand{\cleardoublepage}{
\clearpage\ifodd\c@page\else
\hbox{}
\vspace*{\fill}
\thispagestyle{empty}
\newpage
\fi}

%----------------------------------------------------------------------------------------
%	COMANDI PERSONALIZZATI
%----------------------------------------------------------------------------------------

\newcommand{\Tau}{\mathcal{T}}
\newcommand{\myNewEmptyPage}{\newpage $ $}
\DeclareMathOperator{\sgn}{sgn}
\newcommand{\degree}{^\circ\text{C}} % SIMBOLO GRADI
\newcommand{\notimplies}{\mathrel{{\ooalign{\hidewidth$\not\phantom{=}$\hidewidth\cr$\implies$}}}}
\renewcommand{\qed}{\tag*{$\blacksquare$}}
\newcommand{\latex}{\LaTeX\xspace}
\newcommand{\tex}{\TeX\xspace}
\newcommand{\questeq}{\overset{?}{=}} % è vero che?
\newcommand{\complementary}{^{\mathrm{C}}}
\newcommand{\transpose}{^{\mathrm{T}}}
\renewcommand{\emptyset}{\varnothing} % simbolo insieme vuoto
\newcommand{\Bot}{\perp \!\!\! \perp} % indipendenza
\renewcommand{\tilde}{\widetilde}
\renewcommand{\hat}{\widehat}
\renewcommand{\theta}{\vartheta}
\newcommand{\boxedText}[1]{\noindent\fbox{\parbox{\textwidth-0.5cm}{#1}}}

\usepackage{mathtools} % Serve per i comandi dopo
\DeclarePairedDelimiter{\abs}{\left\lvert}{\right\rvert}
\DeclarePairedDelimiter{\norm}{\left\lVert}{\right\rVert}
\DeclarePairedDelimiter{\sca}{\left\langle}{\right\rangle}

\newcommand{\fou}[1]{\mathcal{F}\left\{#1\right\}}
\newcommand{\ifou}[1]{\mathcal{F}^{-1}\left\{#1\right\}}
\newcommand{\Mc}{\mathcal{M}}
\newcommand{\E}{\mathbb{E}}
\newcommand{\R}{\mathbb{R}}
\newcommand{\N}{\mathbb{N}}
\newcommand{\C}{\mathbb{C}}
\newcommand{\Z}{\mathbb{Z}}
%\DeclarePairedDelimiter{\fou}{\mathcal{F}}{}
%\DeclarePairedDelimiter{\ifou}{\mathcal{F}^{-1}\left\{}{\right\}}

% Simbolo di integrale barrato

\makeatletter
\newcommand*{\mint}[1]{%
  % #1: overlay symbol
  \mint@l{#1}{}%
}
\newcommand*{\mint@l}[2]{%
  % #1: overlay symbol
  % #2: limits
  \@ifnextchar\limits{%
    \mint@l{#1}%
  }{%
    \@ifnextchar\nolimits{%
      \mint@l{#1}%
    }{%
      \@ifnextchar\displaylimits{%
        \mint@l{#1}%
      }{%
        \mint@s{#2}{#1}%
      }%
    }%
  }%
}
\newcommand*{\mint@s}[2]{%
  % #1: limits
  % #2: overlay symbol
  \@ifnextchar_{%
    \mint@sub{#1}{#2}%
  }{%
    \@ifnextchar^{%
      \mint@sup{#1}{#2}%
    }{%
      \mint@{#1}{#2}{}{}%
    }%
  }%
}
\def\mint@sub#1#2_#3{%
  \@ifnextchar^{%
    \mint@sub@sup{#1}{#2}{#3}%
  }{%
    \mint@{#1}{#2}{#3}{}%
  }%
}
\def\mint@sup#1#2^#3{%
  \@ifnextchar_{%
    \mint@sup@sub{#1}{#2}{#3}%
  }{%
    \mint@{#1}{#2}{}{#3}%
  }%
}
\def\mint@sub@sup#1#2#3^#4{%
  \mint@{#1}{#2}{#3}{#4}%
}
\def\mint@sup@sub#1#2#3_#4{%
  \mint@{#1}{#2}{#4}{#3}%
}
\newcommand*{\mint@}[4]{%
  % #1: \limits, \nolimits, \displaylimits
  % #2: overlay symbol: -, =, ...
  % #3: subscript
  % #4: superscript
  \mathop{}%
  \mkern-\thinmuskip
  \mathchoice{%
    \mint@@{#1}{#2}{#3}{#4}%
        \displaystyle\textstyle\scriptstyle
  }{%
    \mint@@{#1}{#2}{#3}{#4}%
        \textstyle\scriptstyle\scriptstyle
  }{%
    \mint@@{#1}{#2}{#3}{#4}%
        \scriptstyle\scriptscriptstyle\scriptscriptstyle
  }{%
    \mint@@{#1}{#2}{#3}{#4}%
        \scriptscriptstyle\scriptscriptstyle\scriptscriptstyle
  }%
  \mkern-\thinmuskip
  \int#1%
  \ifx\\#3\\\else_{#3}\fi
  \ifx\\#4\\\else^{#4}\fi  
}
\newcommand*{\mint@@}[7]{%
  % #1: limits
  % #2: overlay symbol
  % #3: subscript
  % #4: superscript
  % #5: math style
  % #6: math style for overlay symbol
  % #7: math style for subscript/superscript
  \begingroup
    \sbox0{$#5\int\m@th$}%
    \sbox2{$#5\int_{}\m@th$}%
    \dimen2=\wd0 %
    % => \dimen2 = width of \int
    \let\mint@limits=#1\relax
    \ifx\mint@limits\relax
      \sbox4{$#5\int_{\kern1sp}^{\kern1sp}\m@th$}%
      \ifdim\wd4>\wd2 %
        \let\mint@limits=\nolimits
      \else
        \let\mint@limits=\limits
      \fi
    \fi
    \ifx\mint@limits\displaylimits
      \ifx#5\displaystyle
        \let\mint@limits=\limits
      \fi
    \fi
    \ifx\mint@limits\limits
      \sbox0{$#7#3\m@th$}%
      \sbox2{$#7#4\m@th$}%
      \ifdim\wd0>\dimen2 %
        \dimen2=\wd0 %
      \fi
      \ifdim\wd2>\dimen2 %
        \dimen2=\wd2 %
      \fi
    \fi
    \rlap{%
      $#5%
        \vcenter{%
          \hbox to\dimen2{%
            \hss
            $#6{#2}\m@th$%
            \hss
          }%
        }%
      $%
    }%
  \endgroup
}

%----------------------------------------------------------------------------------------
%	APPENDICE
%----------------------------------------------------------------------------------------

\usepackage[toc,page]{appendix}

%----------------------------------------------------------------------------------------
%	SIMBOLI CARTE DA GIOCO
%
% Sono i seguenti:
%   \varheartsuit
%   \vardiamondsuit
%   \clubsuit
%   \spadesuit
%----------------------------------------------------------------------------------------

\DeclareSymbolFont{extraup}{U}{zavm}{m}{n}
\DeclareMathSymbol{\varheartsuit}{\mathalpha}{extraup}{86}
\DeclareMathSymbol{\vardiamondsuit}{\mathalpha}{extraup}{87}

%----------------------------------------------------------------------------------------

\definecolor{grey245}{RGB}{245,245,245}

\newtheoremstyle{blacknumbox} % Theorem style name
{0pt}% Space above
{0pt}% Space below
{\normalfont}% Body font
{}% Indent amount
{\bf\scshape}% Theorem head font --- {\small\bf}
{.\;}% Punctuation after theorem head
{0.25em}% Space after theorem head
{\small\thmname{#1}\nobreakspace\thmnumber{\@ifnotempty{#1}{}\@upn{#2}}% Theorem text (e.g. Theorem 2.1)
%{\small\thmname{#1}% Theorem text (e.g. Theorem)
\thmnote{\nobreakspace\the\thm@notefont\normalfont\bfseries---\nobreakspace#3}}% Optional theorem note

% Per gli unnumbered tolgo il \nobreakspace subito dopo {\small\thmname{#1} perché altrimenti c'è uno spazio tra Teorema e il ".", lo spazio lo voglio solo se sono numerati per distanziare Teorema e "(2.1)"
\newtheoremstyle{unnumbered} % Theorem style name
{0pt}% Space above
{0pt}% Space below
{\normalfont}% Body font
{}% Indent amount
{\bf\scshape}% Theorem head font --- {\small\bf}
{.\;}% Punctuation after theorem head
{0.25em}% Space after theorem head
{\small\thmname{#1}\thmnumber{\@ifnotempty{#1}{}\@upn{#2}}% Theorem text (e.g. Theorem 2.1)
%{\small\thmname{#1}% Theorem text (e.g. Theorem)
\thmnote{\nobreakspace\the\thm@notefont\normalfont\bfseries---\nobreakspace#3}}% Optional theorem note

\newcounter{dummy} 
\numberwithin{dummy}{chapter}

\theoremstyle{blacknumbox}
\newtheorem{definitionT}[dummy]{Definition}
\newtheorem{theoremT}[dummy]{Theorem}

\theoremstyle{unnumbered}
\newtheorem*{ossT}{Remark}
\newtheorem*{dimT}{Proof}

\RequirePackage[framemethod=default]{mdframed} % Required for creating the theorem, definition, exercise and corollary boxes

% orange box
\newmdenv[skipabove=7pt,
skipbelow=7pt,
rightline=false,
leftline=true,
topline=false,
bottomline=false,
linecolor=orange,
backgroundcolor=orange!5,
innerleftmargin=5pt,
innerrightmargin=5pt,
innertopmargin=5pt,
leftmargin=0cm,
rightmargin=0cm,
linewidth=2pt,
innerbottommargin=5pt]{oBox}

% green box
\newmdenv[skipabove=7pt,
skipbelow=7pt,
rightline=false,
leftline=true,
topline=false,
bottomline=false,
linecolor=green,
backgroundcolor=green!5,
innerleftmargin=5pt,
innerrightmargin=5pt,
innertopmargin=5pt,
leftmargin=0cm,
rightmargin=0cm,
linewidth=2pt,
innerbottommargin=5pt]{gBox}

% blue box
\newmdenv[skipabove=7pt,
skipbelow=7pt,
rightline=false,
leftline=true,
topline=false,
bottomline=false,
linecolor=blue,
backgroundcolor=blue!5,
innerleftmargin=5pt,
innerrightmargin=5pt,
innertopmargin=5pt,
leftmargin=0cm,
rightmargin=0cm,
linewidth=2pt,
innerbottommargin=5pt]{bBox}

% dim box
\newmdenv[skipabove=7pt,
skipbelow=7pt,
rightline=false,
leftline=true,
topline=false,
bottomline=false,
linecolor=black,
backgroundcolor=grey245!0,
innerleftmargin=5pt,
innerrightmargin=5pt,
innertopmargin=5pt,
leftmargin=0cm,
rightmargin=0cm,
linewidth=2pt,
innerbottommargin=5pt]{dimBox}

\newenvironment{theorem}{\begin{gBox}\begin{theoremT}}{\end{theoremT}\end{gBox}}	
\newenvironment{definition}{\begin{bBox}\begin{definitionT}}{\end{definitionT}\end{bBox}}	
\newenvironment{oss}{\begin{oBox}\begin{ossT}}{\end{ossT}\end{oBox}}
\newenvironment{dimostrazione}{\begin{dimBox}\begin{dimT}}{\[\qed\]\end{dimT}\end{dimBox}}

%----------------------------------------------------------------------------------------
%	INDICE INIZIALE
%----------------------------------------------------------------------------------------

\setcounter{secnumdepth}{3} % DI DEFAULT LE SUBSUBSECTION NON SONO NUMERATE, COSÌ SÌ
\setcounter{tocdepth}{2} % FISSA LA PROFONDITÀ DELLE COSE MOSTRATE NELL'INDICE

\usepackage[hidelinks]{hyperref} % Rende l'indice interattivo e hidelinks nasconde il bordo rosso dai riferimenti

\usepackage{glossaries} % certain packages that must be loaded before glossaries, if they are required: hyperref, babel, polyglossia, inputenc and fontenc
\setacronymstyle{long-short}

% Questo comando e quello dopo servono per avere il comando \tocless da mettere prima di una sezione che non voglio far apparire nell'indice
\newcommand{\nocontentsline}[3]{}
\newcommand{\tocless}[2]{\bgroup\let\addcontentsline=\nocontentsline#1{#2}\egroup}

%\usepackage{showframe}
\usepackage[textsize=tiny, textwidth=1.5cm]{todonotes} %add disable to options to not show in pdf

%\usepackage[nomarginpar,margin=1in]{geometry}

\usepackage[paperwidth=210mm,
            paperheight=297mm,
            left=1in,
            %top=50pt,
            textwidth=400pt,
            marginparsep=25pt,
            marginparwidth=1in,
            %textheight=692pt,
            footskip=50pt
            ]
           {geometry}


% Comandi pratici


\newcommand{\RR}{\mathbb{R}}

% d nell'integrale e i rispettivi usi
\newcommand{\de}{\,\mathrm d}
\newcommand{\dx}{\de x}
\newcommand{\dy}{\de y}
\newcommand{\dl}{\de l}
\newcommand{\dr}{\de r}
\newcommand{\ds}{\de s}
\newcommand{\dt}{\de t}
\newcommand{\dv}{\de v}
\newcommand{\dxi}{\de \xi}
\newcommand{\drho}{\de \rho}

% d nell'integrale con differenziale vettoriale
\newcommand{\dxx}{\de \x}
\newcommand{\dyy}{\de \y}
\newcommand{\dsig}{\de \sigg}

\allowdisplaybreaks[4] % Consente di rompere equazioni su più pagine

\newacronym{sp}{SP}{stochastic process}
\newacronym{ssp}{SSP}{stationary stochastic process}
\newacronym{mse}{MSE}{mean square error}
\newacronym{wn}{WN}{White Noise}
\newacronym{pem}{PEM}{Prediction Error Minimization}
\DeclareMathOperator{\WN}{WN}
\DeclareMathOperator*{\argmax}{arg\,max}
\DeclareMathOperator*{\argmin}{arg\,min}

% \usepackage{xifthen}
% 1 parametro necessario, con valore "nullo" di default (le due quadre vuote)
% non mettere le seconde quadre equivale a dire che è obbligatorio
% \newcommand{\ar}[1][]{\ifthenelse{\isempty{#1}}{AR}{AR($#1$)}}
% \newcommand{\ma}[1][]{\ifthenelse{\isempty{#1}}{MA}{MA($#1$)}}
% \newcommand{\arma}[1][]{\ifthenelse{\isempty{#1}}{ARMA}{ARMA($#1$)}}
% \newcommand{\arx}[1][]{\ifthenelse{\isempty{#1}}{ARX}{ARX($#1$)}}

%%%%%%%%%%%%%%%%%%%%%%%%%%%%%%%%%%%%%%%%%%%%%%%
%%%%%%%%%%%%%%%%%%%%%%%%%%%%%%%%%%%%%%%%%%%%%%%

\begin{document}

%%%%%%%%%%%%%%%%%%%%%%%%%%%%%%%%%%%%%%%%%%%%%%%
%%%%%%%%%%%%%%%%%%%%%%%%%%%%%%%%%%%%%%%%%%%%%%%

\frontmatter
\pagestyle{empty}
\vspace*{\fill}
\begin{center}
	{\large \textsc{Lecture Notes of}}\\
	\vspace*{0.4cm}
	{\Huge \textsc{Model Identification}}\\
	\vspace*{0.4cm}
	{\Huge \textsc{and Data Analysis}}\\
	\vspace*{1cm}
	{\large {From Professor Simone Garatti's lectures}}\\
	\vspace*{0.1cm}
	{\large for the MSc in Mathematical Engineering}\\
	\vspace*{0.4cm}
	{\large {by Teo Bucci, Filippo Cipriani \& Gabriele Corbo}}\\
	\vspace*{1cm}
	Politecnico di Milano\\A.Y. 2021/2022
\end{center}
\vspace*{\fill}
\newpage

%%%%%%%%%%%%%%%%%%%%%%%%%%%%%%%%%%%%%%%%%%%%%%%
%%%%%%%%%%%%%%%%%%%%%%%%%%%%%%%%%%%%%%%%%%%%%%%

{\Large \textit{Lecture Notes of Model Identification and Data Analysis}}

\vspace*{\fill}

\textcopyright \ Gli autori. Alcuni diritti riservati

Questa opera è rilasciata sotto licenza Creative Commons BY-NC-SA 4.0.\\
\url{https://creativecommons.org/licenses/by-nc-sa/4.0/}

In particolare, senza il permesso degli autori non è consentito fare copie di questo libro (né cartacee né digitali) per rivenderle.

Il codice sorgente \LaTeX \ è disponibile su \\
\url{https://github.com/teobucci/edp-analitica}

\vspace*{1cm}

Revision of \today

Developed by\\
Teo Bucci - \texttt{teo.bucci@mail.polimi.it}\\
Filippo Cipriani - \texttt{filippo.cipriani@mail.polimi.it}\\
Gabriele Corbo - \texttt{gabriele.corbo@mail.polimi.it}\\ \\
Compiled with \ensuremath\heartsuit \\

%\textbf{Prefazione}

Please notify errors or changes through an email or a pull request.

\newpage

%%%%%%%%%%%%%%%%%%%%%%%%%%%%%%%%%%%%%%%%%%%%%%%
%%%%%%%%%%%%%%%%%%%%%%%%%%%%%%%%%%%%%%%%%%%%%%%

% INDICE
\addtocontents{toc}{\protect\thispagestyle{empty}}
\tableofcontents
%\newpage

%%%%%%%%%%%%%%%%%%%%%%%%%%%%%%%%%%%%%%%%%%%%%%%
%%%%%%%%%%%%%%%%%%%%%%%%%%%%%%%%%%%%%%%%%%%%%%%

% PAGINA VUOTA PER FAR PARTIRE IL CAPITOLO IN UNA PAGINA DISPARI
%\myNewEmptyPage

\AtEndDocument{\cleardoublepage}

%%%%%%%%%%%%%%%%%%%%%%%%%%%%%%%%%%%%%%%%%%%%%%%
%%%%%%%%%%%%%%%%%%%%%%%%%%%%%%%%%%%%%%%%%%%%%%%

\mainmatter
\pagestyle{fancy} % Riswitcha per riavere il numero pagina
%\setcounter{page}{1} % Fa ripartire il contatore pagina da 1

%%%%%%%%%%%%%%%%%%%%%%%%%%%%%%%%%%%%%%%%%%%%%%%
%%%%%%%%%%%%%%%%%%%%%%%%%%%%%%%%%%%%%%%%%%%%%%%

\tikzstyle{block}      = [draw, rectangle, inner sep=6pt]
\tikzstyle{every node} = [font=\small]
\tikzstyle{sum}        = [draw, circle, inner sep=6pt]
%\tikzstyle{input} = [coordinate]
%\tikzstyle{output} = [coordinate]
%\tikzstyle{pinstyle} = [pin edge={to-,thin,black}]

\part{MIDA I}

\input{lectures/2022_02_21}
%!TEX root = ../main.tex
\chapter{Stochastic Processes and Model Classes}
A \gls{sp} is an infinite sequence of random variables all defined on the same probabilistic space $(\Omega,\mathcal{A},\mathbb{P})$:
\[
	\ldots,v(1,s),v(2,s),v(3,s),\ldots,v(t,s),\ldots
\]
with $s$: random experiment realization and $t=0,\pm 1,\pm 2,\ldots$: time index.

\begin{obs}
	\gls{sp} extends the notion of random vector (\gls{sp} is a random vector with infinite entries).
\end{obs}

\begin{obs}
For a \emph{fixed} value of the random experiment $s = \overline{s}$, the \gls{sp} becomes the numeric sequence:
\[
	\ldots,v(1,\overline{s}),v(2,\overline{s}),v(3,\overline{s}),\ldots,v(t,\overline{s}),\ldots
\]
which is called \textbf{realization} of the \gls{sp}.
For different values of $s$, one gets different realizations of the \gls{sp}.
\end{obs}

We will think of available observations ${u(1),u(2),\ldots,u(N)}$ and ${y(1), y(2),\ldots, y(N)}$ as finite length realizations.

\textbf{Mean value} $m(t)$: it is the expected value of random variable $v(t,s)$ at time $t$:
\[
	m(t)=\E[v(t, s)]=\int_{\Omega} v(t, s) \mathbb{P}(ds)
\]
$m(t)$ returns the value around which the process take value at time $t$.

\textbf{Covariance function} $\gamma(t_{1}, t_{2})$: it is the expected value of the product of unbiased random variables $(v(t, s)-m(t))$ at two time instants $(t_{1}, t_{2}):$
\begin{align*}
	\gamma(t_{1}, t_{2}) &=\E[(v(t_{1}, s)-m(t_{1}))(v(t_{2}, s)-m(t_{2}))] \\
	&=\int_{\Omega}(v(t_{1}, s)-m(t_{1}))(v(t_{2}, s)-m(t_{2})) \mathbb{P}(ds)
\end{align*}
$\gamma(t_{1}, t_{2})$ quantifies the relation existing between the \emph{gaps} between the process realizations and the mean value at two different time instants.

Particular case: $t_{1}=t_{2}=t$.
\[
	\gamma(t, t)=\E[(v(t, s)-m(t))^{2}]=\int_{\Omega}(v(t, s)-m(t))^{2} \mathbb{P}(ds)
\]
is called the process \textbf{variance function} (it quantifies the process dispersion around its mean value at each time instant).

\section{Stationary Stochastic Processes}
A stochastic process is called \textbf{\gls{ssp}} (wide-sense) if:
\begin{itemize}
	\item $m(t)=m, \forall t$;
	\item $\gamma(t_{1}, t_{2})$ only depends on $\tau=t_{1}-t_{2}$,\\
	i.e. $\gamma(t_{1}, t_{2})=\gamma(t_{3}, t_{4})$ if $t_{1}-t_{2}=t_{3}-t_{4}=\tau, \forall t_{1}, t_{2}, t_{3}, t_{4}$.
\end{itemize}
Idea: the probabilistic properties of a \gls{ssp} are time-translation invariant.

\glspl{ssp} admit a \emph{simplified} representation of the covariance function:
\[
	\boxed{\gamma(\tau)=\gamma(t, t-\tau)=\E[(v(t)-m)(v(t-\tau)-m)]}
\]
where
\[
	\boxed{\gamma(0)=\E[(v(t)-m)^{2}]=\lambda^2} \quad \text{is the variance of the process}
\]
Why \glspl{ssp}?
\begin{itemize}
	\item \emph{Stationary} means \emph{time-invariant} data generating system (situation often encountered in practice).
	\item \gls{ssp} are easier to study.
	\item Non-stationary processes can be recast in the framework of \gls{ssp} by first eliminating the non-stationary part from data (data pre-processing).
\end{itemize}

\textbf{Properties of the covariance function for a \gls{ssp}.}
\begin{itemize}
	\item $\gamma(0)=\E[(v(t)-m)^{2}] \geq 0$ (non negative at initial time)
	\item $|\gamma(\tau)| \leq \gamma(0)$ (bounded)
	\item $\gamma(\tau)=\gamma(-\tau)$ (symmetric) indeed
	\begin{align*}
		\gamma(-\tau)&=\E[(v(t)-m)(v(t-(-\tau))-m)]\\
		&=\E[(v(t)-m)(v(t+\tau))-m)]\\
		&=\E[(v(t+\tau)-m)(v(t))-m)]\\
		&=\gamma(\tau) \quad(t+\tau-t=\tau)
	\end{align*}
\end{itemize}

\fg{0.7}{Screen Shot 2022-03-06 at 00.50.16}

\begin{obs}
\begin{itemize}
	\item Given a \gls{ssp} $x(t)$, we will write $m_{x}$ e $\gamma_{x}(\tau)$ for its mean and covariance function
	\item Two \gls{ssp} $y_{1}(t)$ and $y_{2}(t)$ are wide-sense equivalent if $m_{y_{1}}=m_{y_{2}}$ e $\gamma_{y_{1}}(\tau)=\gamma_{y_{2}}(\tau), \forall \tau$
	\item The \emph{covariance function}
	$$
		\E[(v(t)-m) \cdot(v(t-\tau)-m)]
	$$
	is very \emph{different} from the $2^{\text{nd}}$ order moment function $\E[v(t) \cdot v(t-\tau)]$.
\end{itemize}
\end{obs}
\begin{example}
$\boxed{v(t,s)=\alpha (s)}$, where $\alpha (s)\sim \mathcal{N}(1,3)$.

\begin{itemize}
	\item $m_{v}(t)=\E[v(t, s)]=\E[\alpha(s)]=1=m_{v}$\\
	doesn't depend on $t$;
	\item $\begin{aligned}[t]
		\gamma_{v}(t, t-\tau)&=\E[(v(t, s)-m_{v}(t))(v(t-\tau, s)-m_{v}(t-\tau))]\\
		&=\E[(\alpha(s)-1)(\alpha(s)-1)]=3=\gamma_{v}(\tau)
	\end{aligned}$\\
	doesn't depend on $t$.\\
	Then the process is stationary.
\end{itemize}
\end{example}

\begin{example}
$\boxed{v(t, s)=t \cdot \alpha(s)-t}$, where $\alpha(s) \sim \mathcal{N}(1,3)$.
\begin{itemize}
	\item $m_{v}(t)=\E[v(t, s)]=\E[t \cdot \alpha(s)-t]=t \cdot \E[\alpha(s)]-t=t-t=0$\\
	doesn't depend on $t$;
	\item $\begin{aligned}[t]
		\gamma_{v}(t, t-\tau)&=\E[(v(t, s)-m_{v}(t))(v(t-\tau, s)-m_{v}(t-\tau))]\\
	&=\E[(t \cdot \alpha(s)-t)((t-\tau) \cdot \alpha(s)-(t-\tau))]\\
	&=\E[t \cdot(t-\tau)(\alpha(s)-1)^{2}]\\
	&=t \cdot(t-\tau) \cdot \E[(\alpha(s)-1)^{2}]=t \cdot(t-\tau) \cdot 3
	\end{aligned}$\\
	\emph{does} depend on $t$.\\
	Then the process is not stationary.
\end{itemize}
\end{example}

\begin{obs}
If $\gamma(t, \tau)>0$ then there is a tendency of preserving the sign going from $t$ to $\tau $. The opposite otherwise.
\end{obs}
\section{White Noise}

An \gls{ssp} $e(t)$ is called \textbf{\gls{wn}} with mean $\mu$ and variance $\lambda^{2}$, we shall write
\[
	\boxed{e(t) \sim \WN(\mu, \lambda^{2})}
\]
if the following conditions hold:
\begin{itemize}
	\item $\E[e(t)]=\mu \quad \forall t$
	\item $\gamma_{e}(0)=\E[(e(t)-\mu)^{2}]=\lambda^{2} \quad \forall t$
	\item $\gamma_{e}(\tau)=\E[(e(t)-\mu) \cdot(e(t-\tau)-\mu)]=0 \quad \forall t, \forall \tau \neq 0$
\end{itemize}

The last property is the fundamental one. It says that there is complete incorrelation between random variables at different time instants. The realizations of $e(t)$ are erratic and unpredictable (\textbf{whiteness property}).

%\fg{0.7}{Screen Shot 2022-03-08 at 09.53.35}
\begin{figure}[htpb]
	\centering
	\begin{tikzpicture}
		\draw [-stealth] (-4,0) -- (4,0) node [at end,below] {$\tau$};
		\foreach \x in {-3,-2,-1,1,2,3}{
		    \node [circle,inner sep=1.5pt,fill=black,label=below:{$\x$}] at (\x,0) {};
		}
		\node [below] at (0,0) {$0$};
		\draw [-stealth] (0,0) -- (0,3) node [at end,right] {$\gamma_e(\tau)$};
		\node [circle,inner sep=1.5pt,fill=black,label=right:{$\lambda^2$}] at (0,2) {};
	\end{tikzpicture}
\end{figure}
\FloatBarrier

\begin{obs}
The probability distribution of each single random variables $e(t,s)$ does not matter and is not made explicit in general (wide-sense description of \gls{ssp}).
It could be Gaussian, uniform, etc. (WGN = White Gaussian Noise, WUN = White Uniform Noise, etc.).
\end{obs}

\begin{obs}
Is a constant realization admissible? Yes, it is, but such a realization is \emph{highly unlikely}.
\end{obs}
\gls{wn} is a sort of \emph{building block} to construct a number of different \glspl{ssp}.

To ease the notation, in the following we will consider \emph{zero mean} \gls{wn}. The extension to the general case presents no conceptual difficulties.

\subsection{MA(\texorpdfstring{$n$}{n}) processes}

Read \emph{Moving Average of order $n$}.

Let $e(t) \sim \WN(0, \lambda^{2})$, and MA process is obtained as
\[
	\boxed{y(t)=c_{0} e(t)+c_{1} e(t-1)+c_{2} e(t-2)+\ldots+c_{n} e(t-n)}
\]
In other words, the output $y(t)$ of a MA process is given by a linear combination of the last $n+1$ past values of the input \gls{wn} $e(t)$.
While $t$ is let vary, the linear combination is made on a sliding window (moving average).

\textbf{Mean.}
\[
	m_{y}(t)=\E[y(t)] = \E[c_{0} e(t)+c_{1} e(t-1)+c_{2} e(t-2)+\cdots+c_{n} e(t-n)] = 0+\cdots+0=m_{y}=0,
\]
hence $m_{y}(t)$ doesn't depend on $t$.


%!TEX root = ../main.tex
\textbf{Variance.} (i.e. covariance when $\tau =0$)
\begin{align*}
	\gamma (0)&=\E[(y(t)-m_{y})(y(t)-m_{y})]=\E[(y(t))^2]\\
	&=\E[(c_{0} e(t)+c_{1} e(t-1)+\ldots+c_{n} e(t-n))^{2}]\\
	&=\E[c_{0}^{2} e(t)^{2}+\ldots+c_{n}^{2} e(t-n)^{2}\\
	&\qquad+2 c_{0} c_{1} e(t) e(t-1)+\ldots+2 c_{n-1} c_{n} e(t-n-1) e(t-n)]\\
	&=c_{0}^{2} \E[e(t)^{2}]+c_{1}^{2} \E[e(t-1)^{2}]+\ldots+c_{n}^{2} \E[e(t-n)^{2}]\\
	&\qquad+2 c_{0} c_{1} \E[e(t) e(t-1)]+\ldots+2 c_{n-1} c_{n} \E[e(t-n-1) e(t-n)]
\end{align*}

Since $e(t) \sim \WN(0, \lambda^{2})$, we have that:
$$
\E[e(t)^{2}]=\E[e(t-1)^{2}]=\ldots=\E[e(t-n)^{2}]=\lambda^{2}
$$
and that
$$
\E[e(t) e(t-1)]=\ldots=\E[e(t-n-1) e(t-n)]=0
$$
thus
\[
	\boxed{\gamma (t,0)=\gamma (0)=(c_{1}^2 +c_{1}^2 +\cdots+c_{n}^2 )\cdot\lambda^2}
\]
hence $\gamma (0)$ doesn't depend on $t$.

\textbf{Covariance.}

To calculate the generic covariance, let us proceed with $\tau =1$.
\begin{align*}
	\gamma(t, t-1)&=\E[(y(t)-m_{y})(y(t-1)-m_{y})]\\
	&=\E[y(t) y(t-1)]\\
	&=\E[(c_{0} e(t)+c_{1} e(t-1)+\cdots+c_{n} e(t-n))\cdot (y(t-1))]\\
	&=\E[(c_{0} e(t)+c_{1} e(t-1)+\cdots+c_{n} e(t-n))\cdot (c_{0} e(t-1)+\cdots+c_{n-1} e(t-n)+c_{n} e(t-n-1))]
\end{align*}

Only those terms where the \gls{wn} is multiplied by itself at the same time instant are non null.
\[
	\gamma(t, t-1)=\gamma (1)=(c_{0}c_{1}+c_{1}c_{2}+\cdots+c_{n-1}c_{n})\cdot\lambda^2
\]
hence $\gamma (1)$ doesn't depend on $t$.

Similarly
\begin{align*}
	\gamma (t,t-2)=\gamma (2) &= (c_{0}c_{2}+c_{1}c_{3}+\cdots+c_{n-2}c_{n})\cdot\lambda^2\\
	&\vdots\\
	\gamma (t,t-n)=\gamma (n) &= (c_{0}c_{n})\cdot\lambda^2\\
	\gamma (t,t-n-1)=\gamma (n+1) &= 0
\end{align*}
since all products are uncorrelated. In conclusion
\[
	\gamma (\tau )=\begin{cases}
		(c_{1}^2 +c_{1}^2 +\cdots+c_{n}^2 )\cdot\lambda^2 & \text{if}\ \tau =0\\
		(c_{0}c_{1}+c_{1}c_{2}+\cdots+c_{n-1}c_{n})\cdot\lambda^2 & \text{if}\ \tau =\pm 1\\
		(c_{0}c_{2}+c_{1}c_{3}+\cdots+c_{n-2}c_{n})\cdot\lambda^2 & \text{if}\ \tau =\pm 2\\
		\vdots\\
		(c_{0}c_{n})\cdot\lambda^2 & \text{if}\ \tau =\pm n\\
		0 & \text{if}\ |\tau| > \pm n\\
	\end{cases}
\]
\subsection{MA(\texorpdfstring{$\infty$}{infinity}) processes}

\[
	y(t)=c_{0} e(t)+c_{1} e(t-1)+\cdots+c_{i} e(t-i)+\cdots=\sum_{i=0}^{\infty} c_{i} e(t-i) \quad e(t) \sim \WN\left(0, \lambda^{2}\right)
\]
Assumption: $\sum_{i=0}^{\infty} c_{i}^{2}<\infty$ (it guarantees that $y(t)$ is well defined).

\textbf{Mean.}
\[
	m_{y}(t)=\E[y(t)]=\E\left[\sum_{i=0}^{\infty} c_{i} e(t-i)\right]=\sum_{i=0}^{\infty} c_{i} \E[e(t-i)]=\sum_{i=0}^{\infty} c_{i} \cdot 0=0
\]
doesn't depend on $t$.

\textbf{Variance.}
\begin{align*}
	\gamma_{y}(0)&=\E[(y(t)-m_{y})^{2}]\\
	&=\E\left[\sum_{i=0}^{\infty} c_{i} e(t-i) \cdot \sum_{j=0}^{\infty} c_{j} e(t-j)\right]\\
	&=\E\left[\sum_{i, j=0}^{\infty} c_{i} c_{j} \cdot e(t-i) e(t-j)\right]\\
	&=\sum_{i, j=0}^{\infty} c_{i} c_{j} \cdot \E[e(t-i) e(t-j)]\\
	&=\{\text{non null only when }i=j\}\\
	&=\sum_{i=0}^{\infty} c_{i}^2 \cdot\lambda^2 
\end{align*}
doesn't depend on $t$.

\textbf{Covariance.}
\begin{align*}
	\gamma_{y}(t, t-\tau) &=\E[(y(t)-m_{y}) (y(t-\tau)-m_{y})]\\
	&=\E[y(t) y(t-\tau)]\\
	&=\E\left[\sum_{i=0}^{\infty} c_{i} e(t-i) \cdot \sum_{j=0}^{\infty} c_{j} e(t-j-\tau)\right]\\
	&=\E\left[\sum_{i, j=0}^{\infty} c_{i} c_{j} \cdot e(t-i) e(t-j-\tau)\right]\\
	&=\sum_{i, j=0}^{\infty} c_{i} c_{j} \cdot \E[e(t-i) e(t-j-\tau)]\\
	&=\{\text{non null only when }i=j+\tau\}\\
	&=\sum_{j=0}^{\infty} c_{j+\tau}c_{j}\cdot\lambda^2 
\end{align*}
doesn't depend on $t$.

So if $\sum_{i=0}^{\infty} c_{i}^{2}<\infty$ then the MA($\infty $) process is well defined and is a \gls{ssp}.

\textbf{Observation.} MA($\infty$) processes are very general, they almost \emph{cover} the class of \glspl{ssp} (i.e. apart from few exceptions, all \gls{ssp} can be written as MA($\infty$)).

However, MA($\infty$) are difficult to handle since there are infinite coefficients and, moreover, the computation of the covariance function requires the computation of the sum of an infinite series (hard in general).

On the other hand, MA($n$) are too limited, that is why we will look into AR and ARMA models.
\input{lectures/2022_02_24}
%!TEX root = ../main.tex
\section{AR and ARMA processes}

\subsection{AR(Auto Regressive) processes}
A process $y(t)$ is an AR process if it is generated as:
\begin{align*}
	y(t)=a_{1} y(t-1)+a_{2} y(t-2)+\ldots+a_{m} y(t-m)+e(t)
\end{align*}
where $e(t) \approx W N\left(\mu, \lambda^{2}\right)$.

Terminology:

$\begin{array}{ll}a_{1}, a_{2}, \ldots, a_{m} & \text { AR process (model) coefficient } \\ m & \text { process (model) order; } \\ \operatorname{AR}(m) & \text { AR process of order } m .\end{array}$
 
Hence, the output $y(t)$ of an AR process is recursively defined as the linear combination of last $m$ past values of the process itself plus the input $e(t)$ at the same time instant.

\textbf{Observation.} The difference equation generating the AR process admits non-unique solution unless we specify an initial condition. Which solution do we consider as the AR process?

By AR process we mean the solution obtained by taking the initial condition $\mathbf{y}\left(t_{0}\right)=0$ and letting the initial time instant tends to minus infinity, $t_{0} \rightarrow-\infty$ (in short, we will write $\mathbf{y}(-\infty)=0$ )). In other words, the AR process is the steady-state solution.

\textbf{Example}

$y(t)=a y(t-1)+e(t)$, where $e(t) \approx W N\left(\mu, \lambda^{2}\right)(\operatorname{AR}(1)$ process $)$

What is the steady state solution?
$$
\begin{array}{rlr}
	y(t) & =a y(t-1)+e(t)= & (y(t-1)=a y(t-2)+e(t-1)) \\
	& =e(t)+a e(t-1)+a^{2} y(t-2)= & (y(t-2)=a y(t-3)+e(t-2)) \\
	& \cdots & \\
	& =e(t)+a e(t-1)+a^{2} e(t-2)+\ldots+a^{t-t_{0}} y\left(t_{0}\right)= & (y\left(t_{0}\right)=0) \\
	& \cdots & (t_{0} \rightarrow-\infty) \\
	& =e(t)+a e(t-1)+a^{2} e(t-2)+\ldots+a^{n} e(t-n)+\ldots=\sum_{i=0}^{\infty} a^{i} e(t-i)
\end{array}
$$
The steady state solution is an $\mathrm{MA}(\infty)$ process with coefficients: $c_{0}=1, c_{1}=a, c_{2}=a^{2}, \ldots, c_{i}=a^{i}, \ldots$

In general, AR processes are $\mathrm{MA}(\infty)$ processes with coefficients determined by the AR model coefficients by recursively apply the difference equation.

$\mathrm{MA}(\infty)$ processes are well defined if 
\begin{align*}
	\sum_{i=0}^{\infty} \left(c_{i}\right)^2=\sum_{i=0}^{\infty} \left(a^{i}\right)^2< +\infty
\end{align*}
The geometric series converges $\iff a^2\leq 1$.

So if $|a|\leq1$ the steady-state solution is well defined and SSP.

\subsection{ARMA(Auto Regressive Moving Average) processes}

A process $y(t)$ is an ARMA process if it is generated as:
\begin{align*}
	y(t)&=\\
	&=a_{1} y(t-1)+a_{2} y(t-2)+\ldots+a_{m} y(t-m)+\quad &\operatorname{AR}(m) \text{ part}\\ &+c_{0} e(t)+c_{1} e(t-1)+\ldots+c_{n} e(t-n) . \quad &\operatorname{MA}(n) \text{ part}
\end{align*}

where $e(t) \approx W N\left(\mu, \lambda^{2}\right)$.

Again by ARMA process we mean the steady-state solution obtained by letting $\mathbf{y}(-\infty)=0$.
 
 Similarly to AR processes, the steady-state solution is an MA( $\infty)$ process whose coefficients are obtained from the ARMA model coefficients by recursively apply the difference equation.
 
Terminology:

$m \quad$ AR part order

$n \quad$ MA part order

$\operatorname{ARMA}(m, n) \quad$ ARMA process of orders $m$ and $n$

An ARMA process is well defined and stationary under some conditions which are too complicated to verify directly.
We'll see how to solve this problem after introducing the operatorial 
representation of ARMA processes.

\subsection{Operatorial 
	representation of ARMA processes}

\textbf{Definition} (backward and forward shift operators)

The backward shift operator $z^{-1}$ (from the space of discrete-time signals to the same space) is defined as: $z^{-1} x(t)=x(t-1)$.

Similarly, $z$ is the forward shift operator and: $z x(t)=x(t+1)$.

\textbf{Properties of operators $z^{-1}$ and $z$}

$z^{-1}$ and $z$ are linear:

\begin{align*}
	&z^{-1}(a \cdot x(t)+b \cdot y(t))=a \cdot x(t-1)+b \cdot y(t-1) \\
	&z(a \cdot x(t)+b \cdot y(t))=a \cdot x(t+1)+b \cdot y(t+1)
\end{align*}

$z^{-1}$ and $z$ can be recursively applied:

\begin{align*}
	&z^{-1}\left(z^{-1}\left(z^{-1}(x(t))\right)\right)= \\
	&=z^{-1}\left(z^{-1}(x(t-1))\right)=z^{-1}(x(t-2))=x(t-3)= \\
	&=z^{-3} x(t) \text { (compact notation) }
\end{align*}

(similarly for $z$ )

$z^{-1}$ and $z$ can be linearly composed:

\begin{align*}
	&\left(a z^{-1}+b z+c z^{-3}+d z^{2}\right) x(t)= \\
	&=a\left(z^{-1} x(t)\right)+b(z x(t))+c\left(z^{-3} x(t)\right)+d\left(z^{2} x(t)\right)= \\
	&=a x(t-1)+b x(t+2)+c x(t-3)+d x(t+2)
\end{align*}

We can rewrite an ARMA process as follows:
\begin{align*}
	\left(1-a_{1} z^{-1}-a_{2} z^{-2}-\ldots-a_{m} z^{-m}\right) y(t)=\left(c_{0}+c_{1} z^{-1}+\ldots+c_{n} z^{-n}\right) e(t)
\end{align*}

Even more compact notation:
\begin{align*}
	y(t)=\frac{\left(c_{0}+c_{1} z^{-1}+\ldots+c_{n} z^{-n}\right)}{\left(1-a_{1} z^{-1}-a_{2} z^{-2}-\ldots-a_{m} z^{-m}\right)} e(t)=\frac{C(z)}{A(z)} e(t)
\end{align*}
where:
\begin{align*}
	&C(z)=\left(c_{0}+c_{1} z^{-1}+\ldots+c_{n} z^{-n}\right) \\
	&A(z)=\left(1-a_{1} z^{-1}-a_{2} z^{-2}-\ldots-a_{m} z^{-m}\right)
\end{align*}

$\frac{C(z)}{A(z)}$ is called discrete time transfer function and it simply says that $y(t)$ is generated as the steady-state output of a linear operator that receive as input $e(t)$.
%!TEX root = ../main.tex

\section{Composition of transfer functions and output processes}
\subsection{Series}
Given $u(t)$ \gls{sp}, consider
\begin{align*}
	x(t)&=W_{1}(z)u(t)=\frac{C_{1}(z)}{A_{1}(z)}u(t)\\
	y(t)&=W_{2}(z)x(t)=\frac{C_{2}(z)}{A_{2}(z)}x(t)
\end{align*}
%\fg{0.7}{Screen Shot 2022-03-08 at 16.28.23}
\begin{figure}[htpb]
	\centering
	\begin{tikzpicture}
		% place nodes
		\node [block] (w1) at (0,0) {$W_{1}(z)$};
		\node [block,right=2cm of w1] (w2) {$W_{2}(z)$};

		% connect nodes
		\draw [stealth-] (w1.west) -- ++(-2,0) node[midway,above] {$u(t)$};
		\draw [-stealth] (w1.east) -- (w2.west) node[midway,above] {$x(t)$};
		\draw [-stealth] (w2.east) -- ++(2,0)  node[midway,above] {$y(t)$};
	\end{tikzpicture}
\end{figure}
\FloatBarrier

\begin{theorem}
	The process $y(t)$ is the steady state output of a new filter having transfer function $W_{1}(z)\cdot W_{2}(z)$ fed by $u(t)$. That is,
	\[
		y(t)=[W_{1}(z)\cdot W_{2}(z)]u(t)=\frac{C_{1}(z)\cdot C_{2}(z)}{A_{1}(z)\cdot A_{2}(z)}u(t)
	\]
	meaning that $y(t)$ is the solution to the recursive equation:
	\[
		A_{1}(z)\cdot A_{2}(z)\cdot y(t) = C_{1}(z)\cdot C_{2}(z)\cdot u(t).
	\]
\end{theorem}

\subsection{Parallel}
Given $u(t)$ \gls{sp}, consider
\begin{align*}
	y_{1}(t)&=W_{1}(z)u(t)\\
	y_{2}(t)&=W_{2}(z)u(t)\\
	y(t)&=y_{1}(t)+y_{2}(t)=W_{1}(z)u(t)+W_{2}(z)u(t)=[W_{1}(z)+W_{2}(z)]u(t)
\end{align*}
\fg{0.7}{Screen Shot 2022-03-08 at 16.36.03}
\begin{theorem}
	The process $y(t)$ is the steady state output of a new filter having transfer function $W_{1}(z)+W_{2}(z)$ fed by $u(t)$.
\end{theorem}

\section{Poles and zeros}

\textbf{Remark.}
One can always multiply a transfer function by $z^{m}/z^{m}$.

Consider now $W(z)$ a complex-valued transfer function. Then one can identify:
\begin{itemize}
	\item \textbf{zeros}: all $z\in \mathbb{C}$ such that $W(z)=0$.
	\item \textbf{poles}: all $z\in \mathbb{C}$ such that $W^{-1} (z)=0$.
\end{itemize}
When $C(z),A(z)$ are polynomials with \emph{positive powers}, then
\[
	\text{zeros}=\{z:C(z)=0\} \qquad \text{poles}=\{z:A(z)=0\}
\]

\textbf{Example.}
\begin{align*}
y(t) &=e(t)+\frac{1}{2} e(t-1)+\frac{1}{4} e(t-2) =\left(1+\frac{1}{2} z^{-1}+\frac{1}{4} z^{-2}\right) e(t) \\
&=\frac{1+\frac{1}{2} z^{-1}+\frac{1}{4} z^{-2}}{1} \cdot \frac{z^{2}}{z^{2}}\cdot e(t) =\frac{z^{2}+\frac{1}{2} z+\frac{1}{4}}{z^{2}} e(t)
\end{align*}
Poles are $z_{1}=z_{2}=0$.\\
Zeros are $z$ such that $z^{2}+\frac{1}{2} z+\frac{1}{4}=0$
\[
	z_{1,2}=\frac{-\frac{1}{2} \pm \sqrt{\left( \frac{1}{2}  \right) ^2 -4\cdot\frac{1}{4} } }{2} = -\frac{1}{4}\pm i\frac{\sqrt{3} }{4}
\]
%\fg{0.7}{Screen Shot 2022-03-08 at 16.57.50}
\begin{figure}[htpb]
	\centering
	\begin{tikzpicture}

		% Axes:
		\draw [-stealth] (-4,0) -- (4,0) node [above left]  {$\Re$};
		\draw [-stealth] (0,-2.5) -- (0,2.5) node [below right] {$\Im$};

		% pole
		\node[cross out,draw=black] at (0,0) {};

		\draw[dashed] (-1,-1.73) -- (-1,1.73) node[midway, below left]{$-\frac{1}{4}$};
		
		% zeros    
		\draw[dashed] (0,1.73) node[right] {$i \frac{\sqrt{3}}{4}$} --  (-1,1.73) node[solid, fill=white, circle,draw=black] {};
		\draw[dashed] (0,-1.73) node[right] {$-i \frac{\sqrt{3}}{4}$} --  (-1,-1.73) node[solid, fill=white, circle,draw=black] {};

	\end{tikzpicture}
\end{figure}
\FloatBarrier

\begin{definition}
	We say that $W(z)=\frac{C(z)}{A(z)}$ is:
	\begin{itemize}
		\item \textbf{asymptotically stable} if all \emph{poles} are such that $|z|<1$.
		\item \textbf{minimum phase} if all \emph{zeros} are such that $|z|<1$.
	\end{itemize}
\end{definition}

Consider $W(z)$ rational transfer function, $v(t)$ \gls{sp} (input), $y(t)=W(z)v(t)$.

\begin{theorem}
	If
	\begin{itemize}
		\item $v(t)$ is a \gls{ssp};
		\item $W(z)$ is asymptotically stable;
	\end{itemize}
	then $y(t)$ is well-defined and is also \emph{stationary} (is a \gls{ssp}).
\end{theorem}

In the case of ARMA processes, the input is a \gls{wn}, which is stationary by definition. Thus we just need to check the second condition.

In general one can factorize the denominator:
\begin{align*}
	y(t)&=\frac{C(z)}{A(z)}v(t)=\frac{C(z)}{(z-p_{1})(z-p_{2})\cdots(z-p_{m})}v(t)\\
	&=C(z)\cdot\frac{1}{z-p_{1}}\cdot\frac{1}{z-p_{2}}\cdots\frac{1}{z-p_{m}}v(t)\\
	&=\frac{1}{z-p_{m}}\left[ \frac{1}{z-p_{m-1}}\left[ \cdots\frac{1}{z-p_{1}}\left[ C(z)v(t) \right]   \right]   \right]  
\end{align*}
\fg{0.7}{Screen Shot 2022-03-08 at 17.10.36}
%!TEX root = ../main.tex
Now, let us consider the stochastic processes $y(t)$ obtained as output of an asymptotically stable digital filter $F(z)$ fed by a stationary stochastic process $v(t)$ as input, but with a generic initialization (not steady-state output).

\fg{0.4}{Screenshot (17)}

\textbf{Theorem.} There is just one stationary output which corresponds to the steady-state solution. However, if $F(z)$ is asymptotically stable, then all possible outputs obtained for different initialization of the digital filter $F(z)$ tends asymptotically (as $t \rightarrow \infty$ ) to the steady-state solution, i.e. to the stationary output.

\fg{0.7}{Screenshot (18)}

\section{Weak(Wide sense) characterization of AR,ARMA processes}
\textbf{Goal.} Given AR,ARMA process compute mean $m_y$ and covariance fuction $\gamma_y(\tau)$.

%Since the steady-state solution is an MA($\infty$) process we could use such results but are too diffucult

We'll rely on the recursive equation characterizing such processes.

\subsection{AR processes}
\textbf{Example.}

Let us consider the $\operatorname{AR}(1)$ (or equivalently $\operatorname{ARMA}(1,0)$ ) process generating according to:
\begin{align*}
	y(t)=a \cdot y(t-1)+e(t) \quad \text{where} \quad e(t) \sim W N\left(0, \lambda^{2}\right)
\end{align*}

- Is $y(t)$ stationary?

- Compute $m_{y}$ and $\gamma_{y}(\tau)$ for $\tau=0, \pm 1, \pm 2, \ldots$

Operatorial representation for $y(t)$ :

\begin{align*}
	&y(t)=z^{-1} a y(t)+e(t) \\
	&\left(1-z^{-1} a\right) y(t)=e(t) \\
	&y(t)=\frac{1}{1-z^{-1} a} e(t)
\end{align*}

Transfer function with positive powers (to spot out zeroes and poles): $y(t)=\frac{z}{z-a} e(t) \quad$.

There's just one pole: $z=a .$

The process generating system is asymptotically stable if $|a| <1$. 

Since $e(t)$ is a S.S.P. (by definition of white noise), when $|a| <1$ the steady-state output process $y(t)$ is a S.S.P.

In order to compute the mean, start from the time-domain representation and apply expectation to 
both sides:
\begin{align*}
	\mathbb{E}[y(t)]=\mathbb{E}[a \cdot y(t-1)+e(t)]
\end{align*}
and thanks to linearity:
\begin{align*}
	\mathbb{E}[y(t)]=a \cdot \mathbb{E}[y(t-1)]+]+\mathbb{E}[e(t)]
\end{align*}
Thanks to stationarity $\mathbb{E}[y(t)]=\mathbb{E}[y(t-1)]=m_{y}$, so that $m_{y}=a \cdot m_{y}+m_{e}$.
Then:
$$
m_{e}=0 \Rightarrow m_{y}=0
$$

Let us compute $\gamma_{y}(0)=\mathbb{E}\left[\left(y(t)-m_{y}\right)^{2}\right]$

(since $m_{y}=0$, we have that $\gamma_{y}(0)=\mathbb{E}\left[(y(t))^{2}\right]$ )

Start from $y(t)=a \cdot y(t-1)+e(t)$, take the square and apply operator $\mathbb[\cdot]$ to both side:

$$
\mathbb{E}\left[(y(t))^{2}\right]=\mathbb{E}\left[(a \cdot y(t-1)+e(t))^{2}\right]
$$

Thanks to linearity
$$
\gamma_{y}(0)=a^{2} \mathbb{E}\left[y(t-1)^{2}\right]+\mathbb{E}\left[e(t)^{2}\right]+2 a \mathbb{E}[y(t-1) e(t)]
$$
Mid-terms evaluation:

$2 a \mathbb{E}[y(t-1) e(t)]=0$ (we will show this later)

Indeed, by using the MA($\infty$) representation for $y(t-1)$ (AR process)):
$$
y(t-1)=e(t-1)+a \cdot e(t-2)+a^{2} \cdot e(t-3)+a^{3} \cdot e(t-4)+\ldots
$$
we have that:
\begin{align*}
	&\mathbb{E}[e(t) y(t-1)]= \\
	&=\mathbb{E}\left[e(t) \cdot\left(e(t-1)+a \cdot e(t-2)+a^{2} \cdot e(t-3)+a^{3} \cdot e(t-4)+\ldots\right)\right]=0
\end{align*}
(all products give null contribution)


$\mathbb{E}\left[(y(t-1))^{2}\right]=\gamma_{y}(0)$ (thanks to stationarity)

$\mathbb{E}\left[(e(t))^{2}\right]=\lambda^{2}$

Hence, $\gamma_{y}(0)=a^{2} \gamma_{y}(0)+\lambda^{2}$, so:
\begin{align*}
	\gamma_{y}(0)=\frac{\lambda^{2}}{1-a^{2}}
\end{align*} 

Let us compute $\gamma_{y}(1)=\mathbb{E}\left[\left(y(t)-m_{y}\right) \cdot\left(y(t-1)-m_{y}\right)\right]=\mathbb{E}[y(t) y(t-1)]$ (since $m_{y}=0$ )

Start from $y(t)=a \cdot y(t-1)+e(t)$ and multiply both sides for $y(t-1)$.

Apply operator E[.] to both side:

$$\mathbb{E}[y(t) y(t-1)]=\mathbb{E}[(a \cdot y(t-1)+e(t))(y(t-1))]$$

Thanks to linearity:

$$\gamma_{y}(1)=a \cdot \mathbb{E}\left[(y(t-1))^{2}\right]+\mathbb{E}[e(t) y(t-1)]$$

Mid-terms evaluation:

$\mathbb{E}[e(t) y(t-1)]=0 \quad$ (same as before)

$\mathbb{E}\left[(y(t-1))^{2}\right]=\gamma_{y}(0) \quad$ (we have already computed it!)

$$
\gamma_{y}(1)=a \cdot \gamma_{y}(0)=a \cdot \frac{\lambda^{2}}{1-a^{2}}
$$
Similar rationale for $\gamma_{y}(2)$
$$
\gamma_{y}(2)=\mathbb{E}\left\lfloor\left(y(t)-m_{y}\right)\left(y(t-2)-m_{y}\right)\right]=\mathbb{E}[y(t) y(t-2)]
$$
\begin{align*}
	&\mathbb{E}[y(t) y(t-2)]=\mathbb{E}[(a \cdot y(t-1)+e(t))(y(t-2))] \\
	&\gamma_{y}(2)=a \cdot \mathbb{E}[y(t-1) y(t-2)]+\mathbb{E}[e(t) y(t-2)]
\end{align*}

Since $\mathbb{E}[e(t) y(t-2)]=0$ (we will show this later)
$$
\gamma_{y}(2)=a \cdot \gamma_{y}(1)=a^{2} \cdot \frac{\lambda^{2}}{1-a^{2}}
$$

Summary:

\begin{align*}
	&\gamma_{y}(0)=\frac{\lambda^{2}}{1-a^{2}} \\
	&\left\{\begin{array}{l}
		\gamma_{y}(1)=\gamma_{y}(-1)=a \cdot \gamma_{y}(0) \\
		\gamma_{y}(2)=\gamma_{y}(-2)=a \cdot \gamma_{y}(1) \quad \Rightarrow \gamma_{y}(\tau)=a \cdot \gamma_{y}(\tau-1) \text { con }|\tau| \geq 1 \\
		\ldots
	\end{array}\right.
\end{align*}

Recursive expression for $\gamma_{y}(\tau)$
$$
\gamma_{y}(\tau)=a^{\tau} \cdot \frac{\lambda^{2}}{1-a^{2}}
$$
This result has been established for a generic AR($1$) process
Those equations are called "Yule-Walker equations".

Grafical representation:

\fg{0.7}{Screenshot (14)}

\fg{0.7}{Screenshot (16)}

\subsection{ARMA processes}
$$
y(t)=a_{1} y(t-1)+\ldots+a_{m} y(t-m)+c_{0} e(t)+_{\ldots}+c_{n} e(t-n)
$$
where $e(t) \sim W N\left(0, \lambda^{2}\right)$.

\textbf{Mean}
\begin{align*}
	\mathbb{E}[y(t)] &=\mathbb{E}\left[a_{1} y(t-1)+\ldots+a_{m} y(t-m)+c_{0} e(t)+\ldots+c_{n} e(t-n)\right] \\
	&=a_{1} \mathbb{E}[y(t-1)]+\ldots+a_{m} \mathbb{E}[y(t-m)]+c_{0} \mathbb{E}[e(t)]+\ldots+c_{n} \mathbb{E}[e(t-n)]
\end{align*}
$$
m_{y}=a_{1} m_{y}+\ldots+a_{m} m_{y}+c_{0} \cdot 0+\ldots+c_{n} \cdot 0
$$

By asintotical stability we can prove that $(1-a_1-\ldots-a_m)\neq0$

i.e. $m_{y}=0$.

\textbf{Covariance function}
\begin{align*}
	\mathbb{E}\left[y(t)^{2}\right]&=\mathbb{E}\left[\left(a_{1} y(t-1)+\ldots+a_{m} y(t-m)+c_{0} e(t)+\ldots+c_{n} e(t-n)\right)^{2}\right]=\\
	&= a_{1}{ }^{2} \mathbb{E}\left[y(t-1)^{2}\right]+a_{2}{ }^{2} \mathbb{E}\left[y(t-2)^{2}\right]+2 a_{1} a_{2} \mathbb{E}[y(t-1) y(t-2)]+\ldots \\
	&+c_{0}{ }^{2} \mathbb{E}\left[e(t)^{2}\right]+2 a_{1} c_{0} \mathbb{E}[y(t-1) e(t)]+\ldots
\end{align*}

Hence
$$
\gamma_{y}(0)=a_{1}^{2} \gamma_{y}(0)+a_{2}^{2} \gamma_{y}(0)+2 a_{1} a_{2} \gamma_{y}(1)+\ldots
$$

Then
\begin{align*}
	&\mathbb{E}[y(t) y(t-1)]= \\
	&=\mathbb{E}\left[\left(a_{1} y(t-1)+\ldots+a_{m} y(t-m)+c_{0} e(t)+\ldots+c_{n} e(t-n)\right) y(t-1)\right]= \\
	&=a_{1} \mathbb{E}\left[y(t-1)^{2}\right]+\ldots+c_{0} \mathbb{E}[e(t) y(t-1)]+\ldots
\end{align*}

Proceeding this way:
$$
\left\{\begin{array}{l}
	\gamma_{y}(0)=a_{1}^{2} \gamma_{y}(0)+a_{2}^{2} \gamma_{y}(0)+2 a_{1} a_{2} \gamma \\
	\gamma_{y}(1)=a_{1} \gamma_{y}(0)+\ldots+c_{0} \mathrm{E}[e(t) y(t-1)] \\
	\vdots \\
	\gamma_{y}(m-1)=a_{1} \gamma_{y}(m-2)+\ldots
\end{array}\right.
$$

$m$ variables$-m$ linear equations (YULE-WALKER equations for an ARMA process)

Then, $\gamma_{y}(m), \gamma_{y}(m+1), \ldots$ can be recursevely computed from
$$
\gamma_{y}(0), \gamma_{y}(1), \ldots, \gamma_{y}(m-1)
$$
\begin{align*}
	\gamma_{y}(m)&=\mathrm{E}[y(t) y(t-m)]= \\
	&=\mathrm{E}\left[\left(a_{1} y(t-1)+\ldots+a_{m} y(t-m)+c_{0} e(t)+\ldots+c_{n} e(t-n)\right) y(t-m)\right]=\ldots
\end{align*}


\input{lectures/2022_03_03}
\input{lectures/2022_03_07}
%!TEX root = ../main.tex
\section{Non-zero mean ARMA processes}
Consider now $y(t)$ ARMA process generated as the steady-state output of a linear operator $W(z)$ that receive as input $e(t)$, where $e(t) \approx W N\left(\mu, \lambda^{2}\right)$, with $\mu\neq0$.

\textbf{Gain theorem}: the steady-state output is constant and it holds that:
\begin{align*}
	\mathbb{E}[y(t)]=m_{y}=W(z)|_{z=1} \cdot \mu
\end{align*}
Indeed:
\begin{align*}
		\mathbb{E}[y(t)] &=\mathbb{E}\left[a_{1} y(t-1)+\ldots+a_{m} y(t-m)+c_{0} e(t)+\ldots+c_{n} e(t-n)\right] \\
		&=a_{1} \mathbb{E}[y(t-1)]+\ldots+a_{m} \mathbb{E}[y(t-m)]+c_{0} \mathbb{E}[e(t)]+\ldots+c_{n} \mathbb{E}[e(t-n)]
\end{align*}
	$$
	m_{y}=a_{1} m_{y}+\ldots+a_{m} m_{y}+c_{0} \cdot \mu+\ldots+c_{n} \cdot \mu
	$$
	i.e. $m_{y}=\frac{c_{0}+c_{1}+\ldots+c_{n}}{1-a_{1}-\ldots-a_{m}} \cdot \mu=W(1) \cdot \mu$.
	
Define two new processes (\textbf{unbiased} processes)
$$
\left\{\begin{array}{l}
	\tilde{y}(t)=y(t)-m_{y} \quad\forall t\\
	\tilde{e}(t)=e(t)-m_{e} \quad\forall t
\end{array} \rightarrow \begin{array}{l}
	\mathrm{E}[\tilde{y}(t)]=\mathrm{E}[y(t)]-m_{y}=0 \\
	\mathrm{E}[\tilde{e}(t)]=\mathrm{E}[e(t)]-m_{e}=0
\end{array}\right.
$$
\begin{align*}
	\tilde{y}(t)=& y(t)-m_{y}=\\
	=& a_{1} y(t-1)+\ldots+a_{m} y(t-m)+c_{0} e(t)+\ldots+c_{n} e(t-n)-m_{y} \\
	=& a_{1}\left(\bar{y}(t-1)+m_{y}\right)+\ldots+a_{m e}\left(\bar{y}(t-m)+m_{y}\right)+\\
	&+c_{0}\left(\tilde{e}(t)+m_{e}\right)+_{\ldots}+c_{n}\left(\bar{e}(t-n)+m_{e}\right)-m_{y}= \\
	=& a_{1} \tilde{y}(t-1)+\ldots+a_{m} \tilde{y}(t-m)+c_{0} \tilde{e}(t)+\ldots+c_{n} \tilde{e}(t-n) \\
	&\underbrace{-\left(1-a_{1}-\ldots-a_{m}\right) m_{y}+\left(c_{0}+\ldots c_{n}\right) m_{e}} \\
	& \text { This term is null, remember that } m_{y}=\frac{c_{0}+c_{1}+\ldots+c_{n}}{1-a_{1}-\ldots-a_{m e}} m_{e}=W(1) \cdot \mu
\end{align*}
Hence,
$$
\tilde{y}(t)=a_{1} \tilde{y}(t-1)+\ldots+a_{m} \tilde{y}(t-m)+c_{0} \tilde{e}(t)+\ldots+c_{n} \tilde{e}(t-n)
$$
where $\tilde{e}(t)\approx W N\left(0, \lambda^{2}\right)$, is Standard zero mean ARMA process.

$\tilde{y}(t)$ is the steady-state solution to $W(z)=\frac{A(z)}{C(z)}$ (same transfer function as before) fed by $\tilde{e}(t)$.

Moreover,
$$
\gamma_{y}(\tau)=\mathbb{E}\left[\left(y(t)-m_{y}\right) \cdot\left(y(t-\tau)-m_{y}\right)\right]=\mathbb{E}[(\tilde{y}(t)) \cdot(\tilde{y}(t-\tau))]=\gamma_{\tilde{y}}(\tau)
$$

Pay attention, we cannot drop $m_{y} \neq 0 $! ! ! 

$$\quad \gamma_{y}(\tau) \neq \mathrm{E}[y(t) \cdot y(t-\tau)]$$

We can see this method graphically:

\fg{0.6}{Screenshot (19)}

\section{ARMAX(ARX) processes}
ARMA processes with eXogenous input.

%% I/O systems
A process $y(t)$, generated by a remote white noise input $e(t)$ and by an exogenous (measurable) input $u(t)$, is an ARMAX process if:
\begin{align*}
	y(t)&=a_{1} y(t-1)+a_{2} y(t-2)+\ldots+a_{m} y(t-m)+\quad\quad \text{AR(m) part}\\
	&+c_{0} e(t)+c_{1} e(t-1)+\ldots+c_{n} e(t-n)+\quad\quad \text{MA(n) part} \\
	&+b_{0} u(t-k)+b_{1} u(t-k-1)+\ldots+b_{p} u(t-k-p) \quad\quad \mathrm{X}(k, p) \text{ part}
\end{align*}

ARMAX($m,n,p,k$) ARMAX process of orders $m,n,p$ with input delay between input $u(t)$ and ouput $y(t)$ equal to $k$.

ARX=ARMAX($m,p,0$).

Then:
\begin{align*}
	y(t)=& \frac{\left(b_{0}+b_{1} z^{-1}+\ldots+b_{p} z^{-p}\right) z^{-k}}{\left(1-a_{1} z^{-1}-a_{2} z^{-2}-\ldots-a_{m} z^{-m}\right)} u(t)+\\
	&+\frac{\left(c_{0}+c_{1} z^{-1}+\ldots+c_{n} z^{-n}\right)}{\left(1-a_{1} z^{-1}-a_{2} z^{-2}-\ldots-a_{m} z^{-m}\right)} e(t)
\end{align*}
i.e.
$$
y(t)=\frac{B(z) z^{-k}}{A(z)} u(t)+\frac{C(z)}{A(z)} e(t)
$$
where:

 $\begin{array}{ll}\quad & B(z)=\left(b_{0}+b_{1} z^{-1}+\ldots+b_{p} z^{-p}\right) \\ & C(z)=\left(c_{0}+c_{1} z^{-1}+\ldots+c_{n} z^{-n}\right) \\ &A(z) =\left(1-a_{1} z^{-1}-a_{2} z^{-2}-\ldots-a_{m} z^{-m}\right)\end{array}$

Both $\frac{B(z) z^{-k}}{A(z)}$ and $\frac{C(z)}{A(z)}$ are transfer functions.

An ARMAX processes can be seen as the sum of a deterministic part (output of $\frac{B(z) z^{-k}}{A(z)}$ fed by $u(t)$) and a stochastic part (output of $\frac{C(z)}{A(z)}$ fed by $e(t)$,ARMA).

In the ARX process $C(z)=1$.

\section{ANALYSIS IN THE FREQUENCY DOMAIN}
\textbf{Definition.}
The spectral density of a S.S.P. $y(t)$ (also called the \textbf{spectrum} of $y(t)$ ) is defined as:
\begin{align*}
	\Gamma_{y}(\omega)=\sum_{t=-\infty}^{+\infty} \gamma_{y}(\tau) \cdot e^{-j \omega \tau}.
\end{align*}
In other words, $\Gamma_y(\omega)$ is defined as the Fourier transform $\left\{\gamma_{y}(\tau)\right\}$ of the covariance function.

Properties of $\Gamma_{1}(\omega)$:

1. $\Gamma_{y}(\omega)$ is a real function of the real variable $\omega$, $\operatorname{Im}\left(\Gamma_{y}(\omega)\right)=0 \quad \forall \omega \in \Re$

2. $\Gamma_{y}(\omega)$ is a positive function,
$$
\Gamma_{y}(\omega) \geq 0 \quad \forall \omega \in \mathbb{Y}^{\mathrm{r}}
$$

3. $\Gamma_{y}(\omega)$ is a even function,
$$
\Gamma_{y}(\omega)=\Gamma_{y}(-\omega) \quad \forall \omega \in \Re
$$

4. $\Gamma_{y}(\omega)$ is a periodic function with period equal to $2 \pi$.  $\Gamma_{y}(\omega)=\Gamma_{y}(\omega+k \cdot 2 \pi) \quad \forall \omega \in \Re, \forall k \in Z$.

\textbf{Observation}: as a consequence of 3 and 4 , we will plot the spectral density in the interval $[0, \pi]$.

\textbf{Example.} Let us consider $e(t) \sim W N\left(\mu, \lambda^{2}\right)$. 

Covariance function:
\begin{align*}
	\gamma_{e}(\tau)= \begin{cases}\lambda^{2} & \text { if } \tau=0 \\ 0 & \text { if } \tau \neq 0\end{cases}
\end{align*}


Spectral density:
$$
\begin{aligned}
	\Gamma_{e}(\omega) &=\sum_{\tau=-\infty}^{+\infty} \gamma_{e}(\tau) \cdot e^{-j \omega \tau}=\gamma_{e}(0) e^{-j \omega 0}+\gamma_{e}(1) e^{-j \omega}+\gamma_{e}(-1) e^{j \omega}+\cdots=\\
	&=\gamma_{e}(0)=\lambda^{2}
\end{aligned}
$$
White Noises have constant and equal to $\lambda^{2}$ spectral density.

\textbf{Example} (MA(1) process)

$y(t)=e(t)+c \cdot e(t-1), c \in \mathcal{R}$ (real coefficient)
$e(t)\sim W N(0,\lambda^{2} )$

\begin{align*}
	&\gamma_{y}(0)=\left(1^{2}+c^{2}\right) \cdot \lambda^{2}=(1+c^{2})\lambda^{2}\\
	&\gamma_{y}(1)=(1 \cdot c) \cdot \lambda^{2}=c\lambda^{2} \\
	&\gamma_{y}(\tau)=0 \text { when } \tau=\ldots, \pm 3, \pm 4, \ldots .
\end{align*}
Spectral density (via the definition):
\begin{align*}
	&\Gamma_{y}(\omega)=\sum_{\tau=-\infty}^{+\infty} \gamma_{y}(\tau) \cdot e^{-j \omega \tau}= \\
	&\left(\text { only }=\gamma_{y}(0), \gamma_{y}(\pm 1)\left(e^{-j \omega}\right) \text { are not null }\right) \\
	&=\gamma_{y}(0)+\gamma_{y}(1)\left(e^{-j \omega}\right)+\gamma_{y}(-1)\left(e^{+j \omega}\right)+0= \\
	&=1+c^{2}+c\left(e^{-j \omega}+e^{j \omega}\right)
\end{align*}

Euler representation of the exponential
$$
e^{-j \omega}+e^{+j \omega}=\cos (\omega)-j \sin (\omega)+\cos (\omega)+j \sin (\omega)=2 \cos (\omega)
$$

We have:
$$\Gamma_y(\omega)=1+c^{2}+2 c \cos (\omega).$$
which is real, even and periodic with period $2 \pi$.
%!TEX root = ../main.tex
\section{Spectrum of a \glsentrylong{ssp}}
Given $y(t)$ a \gls{ssp}, we can compute:
\begin{itemize}
	\item Fourier transform
	\[
		\Gamma _{y}(\omega)=\fou{\gamma _{y}(\tau )}=\sum_{\tau =-\infty }^{+\infty} \gamma _{y}(\tau ) e^{-j\omega \tau } \qquad \omega \in [0,\pi ]
	\]
	\item Anti-Fourier transform
	\[
		\gamma _{y}(\tau)=\ifou{\Gamma _{y}(\omega)}=\frac{1}{2\pi }\int_{-\pi }^{\pi } \Gamma _{y}(\omega )e^{j\omega \tau }d\omega  
	\]
\end{itemize}

\begin{obs}
If $\tau =0$ we have
\[
	\gamma _{y}(0)=\E[(y(t)-m_{y})^2 ]=\frac{1}{2\pi }\int_{-\pi }^{\pi } \Gamma _{y}(\omega )d\omega 
\]
\end{obs}
\textbf{Remark.}
There is bijective relationship between the spectrum and the covariance function. Namely to describe $y(t)$ one could use $m_{y}$ and $\gamma _{y}(\tau )$ or $m_{y}$ and $\Gamma _{y}(\omega)$. It gives the same information from a different perspective, but it's useful to understand some properties.

\subsection{An alternative interpretation of the spectral density}
Suppose that an \gls{ssp} $y(t)$ is filtered through an (ideal) pass-band filter:
\fg{0.7}{kinchine-wiener}
where $\tilde{y}(t)$ is the filtered output process.
\begin{theorem}[Kinchine--Wiener]
	The spectral density at a fixed $\omega$ equals to the mean energy of process realizations frequency by frequency:
	\[
		\Gamma _{y}(\overline{\omega})=\lim_{\delta  \to 0} \gamma _{\tilde{y}}(0)
	\]
\end{theorem}

\begin{example}
The \gls{wn} spectral density is constant, i.e. \gls{wn} energy is \emph{equally-distributed} all over the frequency domain.
\end{example}

\begin{figure}[htpb]
	\centering
	\begin{tikzpicture}

	    % x axis
	    \draw[-stealth]
	        (-4,0) -- (0,0) node[below] {$0$}
	        ( 0,0) -- (4,0) node[right] {$\omega$};
	    
	    \draw[-stealth] (0,0) -- (0,2) node[left] {$\Gamma_y(\omega)$};
	    
	    % plot
	    \draw[very thick]
	        (-pi,1) -- (0,1) node[below left] {$\lambda^2$}
	        (  0,1) -- (pi,1){};
	    
	    % projections
	    \draw[dashed]
	        (-pi,1) -- (-pi,0) node[below] {$-\pi$}
	        ( pi,1) -- ( pi,0) node[below] {$ \pi$}
	        {};

	\end{tikzpicture}
\end{figure}
\FloatBarrier

\begin{example}
Consider the AR($1$) process $y(t)=ay(t-1)+e(t)$.
\begin{itemize}
	\item if $a>0$ the sign tends to be maintained;
	\item if $a<0$ the sign tends to change.
\end{itemize}
The transfer function is $y(t)=\frac{1}{1-az^{-1} }e(t)$, then
\begin{align*}
	\Gamma _{y}(\omega )&=\left|\frac{1}{1-ae^{-j\omega } }\right|^2 \cdot\underbrace{\Gamma _{e}(\omega )}_{=\lambda^2 } = \frac{1}{1-ae^{-j\omega}}\cdot\frac{1}{1-ae^{j\omega}} \cdot\lambda^2\\
	&=\frac{1}{1+a^2 e^{-j\omega }\cdot e^{j\omega }-ae^{-j\omega}-ae^{j\omega}} \cdot \lambda^2\\
	&=\frac{1}{1+a^2 -a\left( \frac{e^{j\omega}+e^{-j\omega}}{2} \right)\cdot 2}\\
	&=\frac{1}{1+a^2-2a\cos (\omega)} \cdot\lambda^2
\end{align*}
\end{example}
Depending on the values of $a$ we have different behaviors.

If $a>0$ energy is concentrated at low frequencies
\fg[Left: signal. Right: spectrum.]{0.8}{ar-1-frequency-fig1}
If $a<0$ energy is concentrated at larger frequencies
\fg[Left: signal. Right: spectrum.]{0.8}{ar-1-frequency-fig2}

\begin{example}
Consider the ARMA process $y(t)=\frac{C(z)}{A(z)}e(t)$
\[
	\Gamma _{y}(\omega )=\left|\frac{C(e^{j\omega})}{A(e^{j\omega})}\right|^2 \cdot\lambda^2 =\frac{C(e^{j\omega})}{A(e^{j\omega} )} \cdot \frac{C(e^{-j\omega})}{A(e^{-j\omega} )}\cdot\lambda^2 
\]
$y(t)$ has a \emph{rational} spectral density. The reverse is also true:
\[
	y(t) \text{ ARMA} \iff W(z) \text{ rational}
\]
\end{example}
Indeed we have the following result:
\begin{theorem}
	Let $y(t)$ be a \gls{ssp} with rational spectral density.
	Then, there exists a \gls{wn} process $\xi(t)$ with suitable mean and variance and a rational transfer function $W(z)$ such that:
	\[
		y(t)=W(z)\xi(t)
	\]
	i.e. $y(t)$ is an ARMA process.
\end{theorem}

However, the \emph{choice} of $\xi(t)$ and of $W(z))$ \emph{is not unique}.
The \emph{same} process $y(t)$ can be generated according to an infinite number of different ARMA models.

Let $y(t)=W(z)\xi(t)$, where $\xi(t)$ is a $\WN\sim(\mu,\lambda^2)$. Our goal is show how to construct $\tilde{W}(z)$ and $\tilde{\xi}(t)$ such that $\tilde{W}(z)\neq W(z)$ and $\tilde{\xi}(t)\neq \xi(t)$, but $y(t)=\tilde W(z)\tilde\xi(t)$. Let us consider four different cases:
\begin{enumerate}
	\item 
	Let $\alpha \in \R$
	\[
		y(t)=W(z) \xi(t) = \underbrace{\frac{W(z)}{\alpha}}_{\tilde{W}(z)} \cdot \underbrace{\alpha \xi(t)}_{\tilde{\xi}(t)} = \tilde{W}(z)\tilde{\xi}(t)
	\]
	where they are still both respectively a rational transfer function and a \gls{wn}, indeed:
	\begin{align*}
		\E[\tilde{\xi}(t)]&=\E[\alpha \xi(t)]=\alpha \E[\xi(t)]=\alpha \mu\\
		\gamma_{\tilde{\xi}}(\tau)&=\E[(\alpha\xi(t)-\alpha\mu)(\alpha \xi(t-\tau)-\alpha \mu)]=\alpha^2 \E[(\xi(t)-\mu)( \xi(t-\tau)- \mu)] = \alpha^2 \gamma_{\xi}(\tau)
	\end{align*}

	\item 
	Let $n \in \N$
	\[
		y(t)=W(z) \xi(t)=W(z) \cdot \frac{z^{n}}{z^{n}} \xi(t)=\left[z^{n} \cdot W(z)\right] \xi(t-n)=\tilde{W}(z) \tilde{\xi}(t)
	\]
	where they are still both respectively a rational transfer function and a \gls{wn}, indeed:
	\begin{align*}
		\E[\tilde{\xi}(t)]&=\E[\xi(t-n)]=\mu \\
		\gamma_{\tilde{\xi}}(\tau)&=\E[(\xi(t-n)-\mu)(\xi(t-n-\tau)-\mu)]=\gamma_{\xi}(\tau)
	\end{align*}

	\item 
	Let $p \in \C$ such that $|p|<1$
	\[
		y(t)=W(z) \xi(t) = \underbrace{\left[W(z) \cdot \frac{z-p}{z-p}\right]}_{\tilde{W}(z)} \xi(t)=\tilde{W}(z) \xi(t)
	\]
	where they are still both respectively a rational transfer function and a \gls{wn}.

	\item 
	Let $q$ be a zero of $W(z)$ such that $|q|>1$, that is
	\[
		W(z)=W_{1}(z)\cdot(z-q)
	\]
	then
	\begin{align*}
		y(t)&=W(z) \xi(t)=W_{1}(z) \cdot(z-q) \xi(t)=W_{1}(z) \cdot(z-q) \left[ \frac{z-\frac{1}{q}}{z-\frac{1}{q}}\right] \xi(t)= \\
		&=\underbrace{W_{1}(z)\left(z-\frac{1}{q}\right)}_{\tilde{W}(z)}\cdot\underbrace{\frac{z-q}{z-\frac{1}{q}} \xi(t)}_{\tilde{\xi}(t)} = \tilde{W}(z) \tilde{\xi}(t)
	\end{align*}
	Notice that $\tilde{\xi}$ is well-defined and stationary since we multiplied $\xi(t)$ by an asymptotically stable transfer function ($|z|=\left|\frac{1}{q}\right|<1$).\\
	Moreover they are still both respectively a rational transfer function and a \gls{wn}, indeed let us compute the spectral density of $\tilde{\xi}(t)$:
	\begin{align*}
		\Gamma_{\tilde\xi}(\omega) &=\frac{\left|e^{j \omega}-q\right|^{2}}{\left|e^{j \omega}-\frac{1}{q}\right|^{2}} \cdot \lambda^{2}=\frac{\left(e^{j \omega}-q\right)\left(e^{-j \omega}-q\right)}{\left(e^{j \omega}-\frac{1}{q}\right)\left(e^{-j \omega}-\frac{1}{q}\right)} \cdot \lambda^{2}\\
		&=\frac{1+q^{2}-q\left(e^{j \omega}+e^{-j \omega}\right)}{1+\frac{1}{q^{2}}-\frac{1}{q}\left(e^{j \omega}+e^{-j \omega}\right)} \cdot \lambda^{2}=\frac{1+q^{2}-2 q \cos (\omega)}{1+\frac{1}{q^{2}}-\frac{2}{q} \cos (\omega)} \cdot \lambda^{2}\\
		&=q^{2} \frac{\frac{1}{q^{2}}+1-\frac{2}{q} \cos (\omega)}{1+\frac{1}{q^{2}}-\frac{2}{q} \cos (\omega)} \cdot \lambda^{2}=q^{2} \cdot \lambda^{2}
	\end{align*}
	which is always constant and characterises a \gls{wn}; while
	\[
		\boxed{\tilde \xi(t) = \frac{z-q}{z-\frac{1}{q}} \xi(t)} \implies \tilde \xi(t+1)-\frac{1}{q}\tilde \xi(t)=\xi(t+1)-q\xi(t)
	\]
	taking $\E[\cdot]$ on both sides
	\[
		m_{\tilde\xi} - \frac{1}{q} m_{\tilde\xi} = \mu - q\mu \implies m_{\tilde\xi} = \frac{1-q}{1-\frac{1}{q}} \cdot \mu = -q \frac{\cancel{1-\frac{1}{q}}}{\cancel{1-\frac{1}{q}}}=  - q \cdot \mu
	\]
	so in the end we have $\boxed{\tilde\xi(t)\sim \WN(-q\mu,q^2 \lambda^2 )}$.

	The quantity
	\[
		\boxed{\frac{z-q}{z-\frac{1}{q}}}
	\]
	is known as \textbf{all-pass filter}.
\end{enumerate}
%!TEX root = ../main.tex

\begin{theorem}[Spectral Factorization]
	Let $y(t)$ be a \gls{ssp} with rational spectral density. 
	Then, there exists an unique \gls{wn} process $\xi(t)$ with suitable mean and variance and an unique rational transfer function $W(z)$ such that:
	\begin{align*}
		y(t) = W(z)\xi(t)=\frac{C(z)}{A(z)}\xi(t)
	\end{align*}
    and:
    \begin{enumerate}
		\item $C(z)$ and $A(z)$ are monic (i.e. the coefficients of the maximum degree terms of $C(z)$ and $A(z)$ are equal to 1);
		\item $C(z)$ and $A(z)$ have null relative degree;
		\item $C(z)$ and $A(z)$ are coprime (i.e. they have no common factors);
		\item[4a.]the poles of $W(z)$ are such that $|z|< 1$;
		\item[4b.]the zeroes of $W(z)$ are such that $|z|\leq 1$.
    \end{enumerate}
\end{theorem}

When all the four conditions above are satisfied, we will say that $y(t) = W(z)\xi(t)$ is a \textbf{canonical representation} of $y(t)$.

\begin{obs}
Conditions 1, 2, and 3 remove any ambiguity due to the process described in Case 1, Case 2, and Case 3, respectively. 

Condition 4a ensures that $W(z)$ is asymptotically stable so that $y(t)$ is well defined, while 4b removes any ambiguity due to the process described in Case 4.
\end{obs}
\begin{example}
We want to find the canonical representation of:
\begin{align*}
	y(t)=\frac{2+5z^{-1}+2z^{-2}}{z^2+\frac{5}{6}z+\frac{1}{6}}\cdot e(t)  \qquad \text{where } e(t)\sim \WN(0,1)
\end{align*}
We first need to have the same degree and positive powers, which we can achieve by multiplying by a proper factor and consider a new \gls{wn}:
\begin{align*}
	y(t)&=\frac{2+5z^{-1}+2z^{-2}}{z^2+\frac{5}{6}z+\frac{1}{6}}\cdot e(t)=\frac{2+5z^{-1}+2z^{-2}}{z^2+\frac{5}{6}z+\frac{1}{6}}\cdot\frac{z^2}{2}\cdot\frac{2}{z^2}\cdot e(t)=\frac{z^2+\frac{5}{2}z+1}{z^2+\frac{5}{6}z+\frac{1}{6}}\cdot \eta(t)
\end{align*}
where $\eta(t)=2z^{-2}e(t)=2e(t-2)\sim \WN(2\cdot0,4\cdot 1)\sim \WN(0,4)$.\\
We know check if some terms cancel out:
\begin{align*}
	&\text{zeros:}\quad z^2+\frac{5}{2}z+1=(z+2)\left( z+\frac{1}{2} \right)   \implies z_{1,2}=-2,-\frac{1}{2}\\
	&\text{poles:}\quad z^2+\frac{5}{6}z+\frac{1}{6}=\left( z+\frac{1}{3} \right)  \left( z+\frac{1}{2} \right) \implies p_{1,2}=-\frac{1}{3},-\frac{1}{2}
\end{align*}
Then:
\begin{align*}
	y(t)=\frac{(z+2)\cancel{(z+\frac{1}{2})}}{(z+\frac{1}{3})\cancel{(z+\frac{1}{2})}}\cdot \eta(t) 
\end{align*}
We can now make sure that all zeros and poles are strictly inside the unit circle. In this case it is particularly useful to remember the all-pass filter:
\begin{align*}
	y(t)=\frac{z+2}{z+\frac{1}{3}}\cdot\frac{z+\frac{1}{2} }{z+\frac{1}{2}}\cdot\eta(t)=\frac{z+\frac{1}{2}}{z+\frac{1}{3}}\underbrace{\frac{z+2}{(z+\frac{1}{2})}\eta(t)}_{\xi(t)}=\frac{(z+\frac{1}{2})}{(z+\frac{1}{3})}\cdot \xi(t)
\end{align*}
where $\xi(t)\sim \WN(0,4\cdot2^2)\sim \WN(0,16)$. This is our canonical representation.
\end{example}
\chapter{Prediction}
\section{Linear optimal prediction theory (for ARMA processes)}
Let us consider a zero mean ARMA process:
\[
	y(t)=W(z)e(t) \quad \text{where }e(t)\sim \WN(0,\lambda^{2} )
\]

\textbf{Fundamental assumptions:}\label{assumptions-prediction-theory}
\begin{enumerate}
	\item $y(t) = W(z)e(t)$ is a canonical representation;
	\item $W(z)$ is minimum phase (i.e. all zeros are such that $|z|<1$).\footnote{Note that \emph{canonical} implies that all zeros are such that $|z|\leq 1$, but it's not enough, we have to exclude zeros on the unit circle.}
\end{enumerate}
 
\begin{obs}
The first assumption is not much of a hurdle since every ARMA process admits a canonical representation, and if $y(t)$ is not given in its canonical representation one can easily reconstruct it. The reason why ask for it will be clear later. 

The second one is a limiting assumption (mild) that excludes models belonging to a small subclass that can be approximated by other ARMA. For example the process $y(t)=e(t)+e(t-1)$, which can be written as $y(t)=(1+z^{-1})e(t)$ has a zero in $-1$, but it can be approximated with $y(t)=(1+0.999z^{-1})e(t)$.
\end{obs}

\textbf{Problem} ($k$-step prediction.)

Given the observations of the process $y(t)$ up to time $t$:
$$
	\ldots , y(t-101), y(t-100), \ldots , y(t-2), y(t-1), y(t)
$$
predict the future value of the process at time $t + k$.


A predictor is any function of the available information which is used to guess the future value of the process.

$\hat{y}(t + k \mid t) =$ predictor of $y(t + k)$ given the observations up to time $t$.

In general: 
$$\hat{y}(t + k \mid t) = f ( y(t), y(t-1), y(t-2),\ldots)$$
i.e. the predictor is any function of the available observations of the process $y(t)$.

\textbf{Abstraction.} We suppose to have all the observations from $-\infty$ up to $t$ (infinite sequence of observations).

We will \emph{restrict} ourselves to functions $f$ which are \emph{linear}:
\begin{align*}
	\hat{y}(t + k \mid t)=\sum_{i=0}^{\infty}\alpha_i y(t-i)
\end{align*}
where ${\{\alpha_i\}}_{i=0}^\infty$ are parameters to be selected in order to achieve the \emph{best} result.

\textbf{Intuitive Goal.} 
We want a predictor which is good with respect $S$ (of the specific experiment).
\[
	\hat{y}(t + k \mid t)\approx y(t+k)
\]

We need to quantify this intuitive closeness concept be rewritten in some rigorous mathematical notion of distance valid for random variables.

\begin{definition}[\Gls{mse} of prediction]
	\[
		\E[(y(t+k,s)-\hat{y}(t+k\mid t,s))^2]
	\]
\end{definition}

\textbf{Goal:} choose $\alpha_0,\ldots,\alpha_i,\ldots$ in order to minimize the \gls{mse}.
\input{lectures/2022_03_14}
%!TEX root = ../main.tex

In order to solve the optimal linear prediction problem, let us start from a simpler problem, consider an ARMA process: $y(t)=W(z) \cdot e(t)$, where $e(t) \sim \WN(0, \lambda^{2})$.

Consider its MA($\infty)$ representation:
\begin{align*}
	y(t)&=w_{0} e(t)+w_{1} e(t-1)+w_{2} e(t-2)+\cdots=\sum_{i=0}^{\infty} w_{i} e(t-i)\\
	y(t-1)&=\sum_{i=0}^{\infty} w_{i} e(t-1-i)\\
	y(t-2)&=\sum_{i=0}^{\infty} w_{i} e(t-2-i)\\
	&\vdots
\end{align*}
where
\[
	W(z)=\frac{C(z)}{A(z)}=w_{0}+w_{1} z^{-1}+w_{2} z^{-2}+\cdots
\]
We can substitute our predictor:
\begin{align*}
	\hat{y}(t+k \mid t) &=\alpha_{0} y(t)+\alpha_{1} y(t-1)+\cdots \\
	&=\alpha_{0} \sum_{i=0}^{\infty} w_{i} e(t-i)+\alpha_{1} \sum_{i=0}^{\infty} w_{i} e(t-1-i)+\cdots \\
	&=\beta_{0} e(t)+\beta_{1} e(t-1)+\beta_{2} e(t-2)+\cdots\\
	&=\sum_{i=0}^{\infty} \beta_{i} e(t-i)
\end{align*}
Any linear predictor based on past output observations, can be rewritten as a (linear) predictor based on noise measurements up to time $t$, i.e.
\[
	\{\text{linear predictors from output}\} \subseteq \{\text{linear predictors from noise}\}
\]
We will solve the Optimal Predictor problem in the \emph{linear predictors from noise} set since it admits an easy solution, and then we will prove that the inclusion is actually an \emph{equality}.

The problem can be now formulated as follows:
\[
\boxed{
	\begin{gathered}
		\text{find } \beta_{0}, \beta_{1}, \beta_{2}, \ldots \text{ s.t. } \E\left[(y(t+k)-\hat{y}(t+k \mid t))^{2}\right] \text{ is minimum,}\\
		\text{where } \hat{y}(t+k \mid t) = \sum_{i=0}^{\infty} \beta_{i} e(t-i).
	\end{gathered}
	}
\]

$y(t+k)$ admits a MA($\infty$) representation
\begin{align*}
	y(t+k)&= \sum_{i=0}^{\infty} w_{i} e(t+k-i) \\
	&= w_{0} e(t+k)+w_{1} e(t+k-1)+\cdots+w_{k-1} e(t+1)\\
	&\quad + w_{k} e(t) + w_{k+1} e(t-1) + w_{k+2} e(t-2) + \cdots\\
	&=\sum_{j=0}^{k-1} w_{j} e(t+k-j)+\sum_{j=k}^{\infty} w_{j} e(t+k-j) \\
	&=\sum_{j=0}^{k-1} w_{j} e(t+k-j)+\sum_{i=0}^{\infty} w_{k+i} e(t-i)
\end{align*}
Then:
\begin{align*}
	&\E\left[(y(t+k)-\hat{y}(t+k \mid t))^{2}\right]= \\
	&=\E\left[\left(\sum_{j=0}^{k-1} w_{j} e(t+k-j)+\sum_{i=0}^{\infty} w_{k+i} e(t-i)-\sum_{i=0}^{\infty} \beta_{i} e(t-i)\right)^{2}\right]\\
	&=\E\left[\left(\sum_{j=0}^{k-1} w_{j} e(t+k-j)+\sum_{i=0}^{\infty} (w_{k+i}-\beta_{i}) e(t-i)\right)^{2}\right]\\
	&=\E\left[\left(\sum_{j=0}^{k-1} w_{j} e(t+k-j)\right)^{2}+\left(\sum_{i=0}^{\infty} (w_{k+i}-\beta_{i}) e(t-i)\right)^{2}+\right.\\
	&\quad\left.+2\cdot\left(\sum_{j=0}^{k-1} w_{j} e(t+k-j)\right)\left(\sum_{i=0}^{\infty} (w_{k+i}-\beta_{i}) e(t-i)\right)\right]\\
	&= \E\left[\left(\sum_{j=0}^{k-1} w_{j} e(t+k-j)\right)^{2}\right]+\E\left[\left(\sum_{i=0}^{\infty} (w_{k+i}-\beta_{i}) e(t-i)\right)^{2}\right]+\\
	&\quad +2 \cdot \E\left[\left(\sum_{j=0}^{k-1} w_{j} e(t+k-j)\right)\left(\sum_{i=0}^{\infty} (w_{k+i}-\beta_{i}) e(t-i)\right)\right]
\end{align*}
However,
\begin{align*}
	\E\Bigg[\underbrace{\left(\sum_{j=0}^{k-1} w_{j} e(t+k-j)\right)}_{\substack{\text{depends on }\\e(t+k),e(t+k-1),\ldots,e(t+1)}}\underbrace{\left( \sum_{i=0}^{\infty} (w_{k+i}-\beta_{i}) e(t-i)\right)}_{\substack{\text{depends on }\\e(t),e(t-1),\ldots}}\Bigg]=0
\end{align*}
All the resulting products are between uncorrelated terms
$\E[e(t+k-j) e(t-i)]=0$ (recall that we assumed $e(t) \sim \WN(0, \lambda^{2})$, i.e.
noise was zero mean).

Hence:
\begin{align*}
	\E\left[(y(t+k)-\hat{y}(t+k \mid t))^{2}\right]=\E\left[\left(\sum_{j=0}^{k-1} w_{j} e(t+k-j)\right)^{2}\right]+\E\left[\left(\sum_{i=0}^{\infty} (w_{k+i}-\beta_{i}) e(t-i)\right)^{2}\right]
\end{align*}
When we minimize with respect to $\beta_{0}, \beta_{1}, \beta_{2}, \ldots$, the first term cannot be modified, while at best the second term can be made equal to zero by choosing:
\[
	\boxed{\beta_{i} = w_{k+i}}
\]
So the optimal linear predictor based on noise measurements is:
\begin{align*}
	\boxed{\hat{y}(t+k \mid t) = \sum_{i=0}^{\infty}w_{k+i} e(t-i)}
\end{align*}
Notice that $y(t+k)$ can be written as the sum of a function of future, unpredictable from the info at $t$, and a predictable part. 
\begin{align*}
	y(t+k)= \underbrace{\sum_{j=0}^{k-1} w_{j} e(t+k-j)}_{\text{unpredictable at time $t$}} + \underbrace{\sum_{i=0}^{\infty} w_{k+i} e(t-i)}_{\substack{\text{predictable at time $t$}\\\hat{y}(t+k \mid t)}}
\end{align*}

\textbf{How to compute $w_{k+i}$?}

We perform the $k$-steps division between the numerator and the denominator:
%\fg{0.4}{Screen Shot 2022-03-22 at 23.14.44}
\begin{figure}[htpb]
	\centering
	\begin{tikzpicture}

	\draw[-] (0,0) -- (0,4);
	\draw[-] (0,3) -- (2,3);
	\draw[-] (0,1) -- (-2,1);

	\node at (1,3.5) {$A(z)$};
	\node at (1,2.5) {$E(z)$};
	\node at (-1,3.5) {$C(z)$};
	\node at (-1,0.5) {$z^{-k}F(z)$};

	\node at (-1,2.5) {$\vdots$};
	\node at (5,2) {$\boxed{\frac{C(z)}{A(z)}=E(z)+z^{-k} \frac{F(z)}{A(z)}}$};
		
	\end{tikzpicture}
\end{figure}
\FloatBarrier
where:
\begin{align*}
	E(z)&=w_{0}+w_{1} z^{-1}+\cdots+w_{k-1} z^{-k+1} \\
	z^{-k} \frac{F(z)}{A(z)}&=w_{k} z^{-k}+w_{k+1} z^{-k-1}+w_{k+2} z^{-k-2}+\cdots
\end{align*}
Hence,
\begin{align*}
	y(t+k) &=\frac{C(z)}{A(z)} e(t+k)=\left[E(z)+z^{-k} \frac{F(z)}{A(z)}\right] e(t+k)\\
	&=\underbrace{E(z)e(t+k)}_{\text{unpredictable at }t}+\underbrace{\frac{F(z)}{A(z)}e(t)}_{\text{predictable at }t}
\end{align*}
The optimal predictor from noise is given by the predictable part of $y(t + k)$:
\[
	\boxed{\hat{y}(t+k \mid t) = \frac{F(z)}{A(z)}e(t)}
\]
$\hat{y}(t+k \mid t)$ is the steady-state output of a suitable linear filter fed by $e(t)$.

\textbf{Unfortunately this result is completely useless in practice.}

In order to actually compute the predicted value for the output variable based on the above expression for $\hat{y}(t+k \mid t)$, past values of the noise process $e(t),e(t_1),e(t_2),\ldots$ should be accessible, but the output $y(t)$ is the only information in reality which we can access.

In order to use it in practice, we need to express the predicted value as a function of past values of the output variable $y(t), y(t_1), y(t_2),\ldots$.

\textbf{Is it possible to reconstruct the noise $e(t)$ from the output $y(t)$?}

If the transfer function $\frac{C(z)}{A(z)}$ is \emph{canonical} (and thus \emph{asymptotically stable}) and \emph{minimum phase} the answer is \emph{yes}.

Since
\begin{equation}\label{eq:reconstruction-noise-from-data}
	e(t)=W(z)^{-1} y(t)=\frac{A(z)}{C(z)} y(t)=\breve{w}_{0} y(t)+\breve{w}_{1} y(t-1)+\breve{w}_{2} y(t-2)+\cdots = \sum_{j=0}^{\infty} \breve{w}_{j}y(t-j)
\end{equation}
then we can substitute this expression in the predictor
\[
	\boxed{\hat{y}(t+k \mid t)=\sum_{i=0}^{\infty} w_{k+i}\left(\sum_{j=0}^{\infty} \breve{w}_{j}y(t-j-i)\right)}
\]
which is now a predictor form output and it is also optimal.

%Suppose it is not, then $\exists \hat{y}(t+k \mid t)=\sum_{i=0}^{\infty} \tilde{\alpha_i}y(t-i)$ which is better than our predictor from noise. But since the first can be written itself as a predictor from noise this is a contradiction.
%\[
%	\E\left[(y(t+k)-\sum_{i=0}^{\infty} \tilde{\alpha_i}y(t-i))^2\right]<\E\left[(y(t+k)-\hat{y}(t+k \mid t))^{2}\right]
%\]
%!TEX root = ../main.tex

What we did \eqref{eq:reconstruction-noise-from-data} is indeed possible (meaning we can reconstruct $e(t)$ from $y(t)$) thanks to the further assumptions we made at page \pageref{assumptions-prediction-theory}. In fact, this means that $W(z)^{-1}=\frac{A(z)}{C(z)}$ is asymptotically stable too.

From the point of view of transfer functions we have that:
\[
	\hat y (t+k\mid t) = \frac{F(z)}{A(z)}e(t) =\frac{F(z)}{\cancel{A(z)}}\cdot\frac{\cancel{A(z)}}{C(z)}y(t) =\frac{F(z)}{C(z)}y(t) \implies \boxed{\hat y (t+k\mid t) = \frac{F(z)}{C(z)}y(t)}
\]
meaning that the optimal predictor from output is obtained as the output of a digital filter $F(z)/C(z)$ fed by $y(t)$ up to time $t$.

\textbf{Remark.}
The correct expression of the linear predictor can \emph{only} be obtained by using the canonical representation, otherwise there may be zeros of $C(z)$ which becomes unstable poles when reconstructing $e(t)$.

Let us illustrate this fact through an example.

\begin{exa}
Let $e(t)\sim \WN(0,1)$ and consider MA($1$) process:
\begin{align*}
	y(t) &= e(t) - 2 e(t-1)\\
	&= (1-2z^{-1} )e(t) &\text{non-canonical}\\
	&=\left( 1-\frac{1}{2} z^{-1}  \right) \cdot\frac{1-2z^{-1}}{1-\frac{1}{2} z^{-1}} e(t)\\
	&=\left( 1-\frac{1}{2} z^{-1}  \right) \xi(t) &\text{canonical}
\end{align*}
where $\xi(t)\sim \WN(0,4)$. The first form was non-canonical because there was a zero ($z=2$) outside the unit circle.

The original process (non-canonical) can be written as
\[
	y(t+1)=\underbrace{e(t+1)}_{\substack{\text{unpred.}\\\text{at $t$}}}-\underbrace{2e(t)}_{\substack{\text{pred.}\\\text{at $t$}}}
\]
thus our optimal predictor from noise would be
\[
	\hat y^{e} (t+1\mid t) = -2e(t)
\]
Whereas the process in canonical form can be written as 
\[
	y(t+1) =\underbrace{\xi(t+1)}_{\substack{\text{unpred.}\\\text{at $t$}}}-\underbrace{\frac{1}{2} \xi(t)}_{\substack{\text{pred.}\\\text{at $t$}}}
\]
thus our optimal predictor from noise would be
\[
	\hat y^{\xi} (t+1\mid t)=-\frac{1}{2} \xi(t).
\]
The problem now is that if we try to reconstruct the noise from the output in the predictor $\hat y^{e} (t+1\mid t) = -2e(t)$ obtained by the non canonical form, we get:
\begin{align*}
	y(t)=(1-2z^{-1})e(t) \iff & e(t) =\frac{1}{1-2z^{-1}} y(t)\\
	\implies & \hat y^{e} (t+1\mid t) = -2e(t) = -\frac{2}{1-2z^{-1}}y(t)
\end{align*}
this is a \textbf{fatal mistake}, because it's not well defined, $e(t)$ cannot be reconstructed here!

The correct way is indeed using the canonical representation:
\begin{align*}
	y(t) = \left( 1-\frac{1}{2} z^{-1}  \right) \xi(t) \iff & \xi(t) = \frac{1}{1-\frac{1}{2} z^{-1} } y(t)\\
	\implies & \boxed{\hat y^{\xi} (t+1\mid t)= -\frac{1}{2} \xi(t) = \frac{-\frac{1}{2} }{1-\frac{1}{2} z^{-1} } y(t)}
\end{align*}
If we consider the prediction error we see that they are different:
\[
	\begin{rcases}
		\E\left[\left( y(t+1)-\hat    y^{e}(t+1\mid t) \right)^2 \right] = \E[e(t+1)^2]   = 1\\
		\E\left[\left( y(t+1)-\hat y ^{\xi}(t+1\mid t) \right)^2 \right] = \E[\xi(t+1)^2] = 4
	\end{rcases}
	\neq
\]
this means that $\hat y^{e} $ and $\hat y^{\xi} $ cannot be both the canonical predictor from output.
\end{exa}

\begin{rem}
\begin{align*}
	y(t+1)=e(t+1)-2e(t) \iff e(t)&=-\frac{1}{2} y(t+1)+\frac{1}{2} e(t+1)\\
	&= -\frac{1}{2} y(t+1)\underbrace{-\frac{1}{2} y(t+2)+\frac{1}{4}e(t+2)}_{\frac{1}{2} e(t+1)}\\
	&= -\frac{1}{2} y(t+1)-\frac{1}{4}y(t+2)-\frac{1}{8}y(t+3)+\frac{1}{8}e(t+3)\\
	&\quad\vdots
\end{align*}
This term is well-defined because the series converges, however we see that the noise of the non-canonical process cannot be reconstructed from the \emph{past} but from the \emph{future}! This is why it is not suitable to produce a predictor.

If instead we try to get something using values from the past we get a term which diverges:
\begin{align*}
	y(t)=e(t)-2e(t-1) \iff e(t)&=y(t)+2e(t-1)\\
	&=y(t)+\underbrace{2y(t-1)+2e(t-2)}_{2e(t-1)}\\
	&=y(t)+2y(t-1)+4y(t-2)+4e(t-3)\\
	&\quad\vdots
\end{align*}
\end{rem}

\textbf{Remark.}
In the assumptions we made at page \pageref{assumptions-prediction-theory} the really fundamental ones are that all zeros and poles are such that $|z|<1$, without this we cannot proceed; the other requests ($C(z),A(z)$ monic, coprime, null relative degree) allows for simplified formulas.

\textbf{Remark.}
Given the construction of $\hat y(t+k\mid t)$, since it's the output of an asymptotically stable filter ($F/C$) fed by $y(t)$ (which is a \gls{ssp}), also $\hat y(t+k\mid t)$ is a \gls{ssp}. The statistical properties of $\hat y(t+k\mid t)$ don't depend on $t$, thus
\[
	\hat y(t+k\mid t) = \frac{F(z)}{C(z)}y(t) \quad \text{and} \quad \hat y(t\mid t-k) = \frac{F(z)}{C(z)}y(t-k)
\]
are completely equivalent.

\textbf{Remark.}
From what we said so far, in order to compute $\hat y(t+k\mid t)$ one should know:
\[
	y(t),y(t-1),y(t-2),\ldots
\]
but in practice we only have a finite sequence back until $t=1$, the time where we started collecting information:
\[
	y(t),y(t-1),y(t-2),\ldots,y(1)
\]
so usually we use the optimal predictor, but instead of considering the steady-solution, we initialize it with:
\[
	\boxed{\hat y(k\mid 0) = \hat y(k-1\mid -1) = \hat y(k-2\mid -2) = \cdots = 0 = y(0) = y(-1) = y(-2) = \cdots}
\]
Thanks to the asymptotic stability of $F(z)/C(z)$ the effect of the conventional initialization rapidly vanishes, and is negligible provided that $t$ is large enough.

\section{Optimal prediction for non-zero mean ARMA processes}
\[
	y(t)=\frac{C(z)}{A(z)}e(t)\qquad e(t)\sim \WN(\mu,\lambda^2 )
\]
Same assumptions as in page \pageref{assumptions-prediction-theory}. How do we compute $\hat y(t+k\mid t)$?
\[
	\E[e(t)] = \mu \implies \E[y(t)]=\frac{C(1)}{A(1)}\cdot\mu=m_{y} \quad \forall t
\]
We construct the unbiased processes:
\[
	\begin{cases}
		\tilde y(t)=y(t)-m_{y}\\
		\tilde e(t)=e(t)-\mu
	\end{cases}
	\text{then (see section \ref{sec:non-zero-mean-arma})}
	\quad
	\tilde y(t)=\frac{C(z)}{A(z)}\tilde e(t) \quad \tilde e(t)\sim \WN(0,\lambda^2)
\]
so that we have:
\[
	y(t)=\tilde y(t)+m_{y} \quad\text{and}\quad y(t+k)=\tilde y(t+k)+m_{y}.
\]
$\tilde y$ is a zero mean ARMA for which we know the solution:
\[
	\hat{\tilde y} (t+k\mid t) = \frac{F(z)}{C(z)}\tilde y(t) \qquad \frac{C(z)}{A(z)}=E(z)+z^{-k}\frac{F(z) }{A(z)},
\]
However we are not interested in predicting $\tilde y$, we want to predict $y$:
\begin{align*}
	\hat y(t+k\mid t)&=\widehat{\tilde y(t+k)+m_{y}}\\
	&= \hat{\tilde y}(t+k\mid t)+m_{y}\\
	&=\frac{F(z)}{C(z)}\tilde y(t)+m_{y}\\
	&=\frac{F(z)}{C(z)}(y(t)-m_{y})+m_{y}\\
	&=\frac{F(z)}{C(z)}y(t)-\frac{F(z)}{C(z)}m_{y}+m_{y}\\
	&=\frac{F(z)}{C(z)}y(t)+\left( 1-\frac{F(1)}{C(1)} \right)  \cdot m_{y}
\end{align*}
where we used the Gain theorem (see \ref{thm:gain-theorem}). Our final solution is:
\[
	\boxed{\hat y(t+k\mid t) = \frac{F(z)}{C(z)}y(t)+\left( 1-\frac{F(1)}{C(1)} \right) m_{y}}
\]
\input{lectures/2022_03_17}
%!TEX root = ../main.tex

ARMA X
\[
	y(t)=\frac{B(z)}{A(z)} u(t-a)+\frac{C(z)}{A(z)}e(t)\qquad e(t)\sim \WN(0,\lambda^2 )
\]
available information:
\begin{gather*}
	y(t),y(t-1),\ldots \\
	u(t),u(t-1),\ldots
\end{gather*}
%!TEX root = ../main.tex
\section{Optimal prediction of ARMAX processes}

\[
	y(t)=\underbrace{\frac{B(z)}{A(z)} u(t-a)}_{\text{deterministic}}+
	\underbrace{\frac{C(z)}{A(z)} e(t)}_{\text{stochastic}}\qquad e(t)\sim \WN(0,\lambda^2 )
\]
Without loss of generality NON zero mean can be always incorporated in $u$.

available information:
\begin{gather*}
	y(t),y(t-1),\ldots \\
	u(t),u(t-1),\ldots
\end{gather*}

\textbf{Hypothesis:}
\begin{itemize}
	\item $\frac{C(z)}{A(z)}e(t)$  is a canonical representation (otherwise compute it)
	\item either $u$ is completely known(pre deterministic signal) from $t=-\infty$ up to $t=+\infty$ or $d\geq k$ (delay bigger than prediction error).
\end{itemize}

Let 
$$
	z(t)=y(t)-\frac{B(z)}{A(z)} u(t-d)
$$

Then, $z(t)=\frac{C(z)}{A(z)} e(t)$ i.e. it is an ARMA process s.t.:

$$
	\frac{C(z)}{A(z)}=E(z)+z^{-k} \frac{F(z)}{A(z)} \quad\text{($k$-steps division between $C(z)$ and $A(z)$)}
$$
Then:
$$
	\hat{z}(t+k \mid t)=\frac{F(z)}{C(z)} z(t)
$$

<<<<<<< Updated upstream
$$
	y(t)=\frac{B(z)}{A(z)} u(t+k-d)+z(t+k)
$$
The first part is deterministically known, and hence can be trivially predicted.
\begin{align*}
	\hat{y}(t+k \mid t)&=\frac{B(z)}{A(z)} u(t+k-d)+z(t+k \mid t) \\
	&=\frac{B(z)}{A(z)} u(t+k-d)+\frac{F(z)}{C(z)} z(t) \\
	& =\frac{B(z)}{A(z)} u(t+k-d)+\frac{F(z)}{C(z)}\Bigg(y(t)-\frac{B(z)}{A(z)} \underbrace{u(t-d)}_{z^{-k}u(t+k-d)}\Bigg) \\
	&=\frac{B(z)}{C(z)} \cdot\Bigg(\underbrace{\frac{C(z)}{A(z)}-\frac{z^{-k} F(z)}{A(z)}}_{E(z)}\Bigg) u(t+k-d)+\frac{F(z)}{C(z)} y(t)
\end{align*}
$$
	\hat{y}(t+k \mid t) =\frac{B(z) E(z)}{C(z)} \underbrace{u(t+k-d)}_{\text{known}}+\frac{F(z)}{C(z)} y(t)
$$

\section{Model Identification}

Up to now, we considered (ARMA/ARMAX) models and studied their properties: covariance and spectrum computation, prediction.

But where does the model come from?

Model identification: retrieve suitable model from experiments on the real system 

\fg{0.4}{Screenshot (25)}

Identification problem: define an automatic procedure to find a model for $S$ based on available (input/output or time series) data.

%\fg{0.7}{Screenshot (26)}
\begin{figure}[htpb]
	\centering
	\begin{tikzpicture}

	% place nodes
		\node [sum] (sum) at (0,0) {};
		\node [block,above=1cm of sum]  (ca) {$\frac{C(z)}{A(z)}$};
		\node [block,left =1cm of sum]  (ba) {$\frac{B(z)}{A(z)}$};
		%\node [above left = 0cm and 2.5cm of ba] {$\begin{array}{c} y(1),y(2),\ldots,y(N) \\ u(1),u(2),\ldots,u(N) \end{array}\implies$};

		% connect nodes
		\draw[-stealth] (ba.east) -- (sum.west) node[near end,above]{$+$};
		\draw[-stealth] (ca.south) -- (sum.north) node[near end,left]{$+$};
		\draw[-stealth] (sum.east) -- ++(2,0) node[midway,above]{$y(t)$};
		\draw[stealth-] (ca.west) -- ++(-2,0) node[midway, above]{$e(t)$};
		\draw[stealth-] (ba.west) -- ++(-2,0) node[midway, above]{$u(t-d)$};

	\end{tikzpicture}
\end{figure}
\FloatBarrier

\subsection{Parametric model identification}

First we select a parametric model class $\mathcal{M}(\vartheta)$ ($\vartheta$= parameters vector – each different $\vartheta$ corresponds to a different model).

For example:
$$
	\mathcal{M}=\left\{y(t)=\frac{B(z, \vartheta)}{A(z, \vartheta)} u(t-d)+\frac{C(z, \vartheta)}{A(z, \vartheta)} e(t), \quad e(t) \sim WN\left(0, \lambda^{2}\right), \vartheta\in\Theta\right\}
$$
is the family of ARMAX models whose coefficients are polynomjals.
\begin{align*}
	A(z, \vartheta)&=1-a_{1}(\vartheta) z^{-1}-\cdots-a_{m}(\vartheta) z^{-m} \\
	B(z, \vartheta)&=b_{1}(\vartheta)+b_{2}(\vartheta) z^{-1}+\cdots+b_{p}(\vartheta) z^{-p}\\
	C(z, \vartheta)&=c_0(\vartheta)+c_{1}(\vartheta) z^{-1}+\cdots+c_{n}(\vartheta) z^{-n}
\end{align*}

We will talk of \textbf{black-box identification} when no knowledge on the system is available and the model structure must be found from data only.

$\vartheta$ is directly the vector of coefficients of $A(z, \vartheta),B(z, \vartheta),C(z, \vartheta)$.

$$
	\vartheta=[a_{1},\ldots,a_{m},b_{1},\ldots,b_{p},c_{1},\ldots,c_{n}]\transpose
$$

In all other cases: \textbf{grey-box identification}.

We try to incorporate some a priori information about the real $S$ in the model class. $\vartheta$ may have some physical interpretation.

\emph{Example.}
$$
	M(\vartheta): y(t)=\frac{b+b^{2} z^{-1}}{1-a z^{-1}} u(t-d)+\frac{1+a z^{-1}}{1-a z^{-1}} e(t)
$$
Here, the parameter vector is given by $\vartheta=\begin{bmatrix}a \\ b\end{bmatrix}$ only.


\textbf{Observation.}
$\lambda^2$ is a parameter too which needs to be identified, however, $\lambda^2$ is much less important than other parameters. So, we will indicate by $\vartheta$ the vector of \emph{important} parameters and keep $\lambda^2$ aside.

$\Theta$ is the set of admissible values for the parameter vector $\vartheta$.

It incorporates a-priori information on the possible value for the parameters. In the blackbox case $\Theta$ is as free as possible.

As we will see, to perform identification we will rely on the theory of prediction. Hence, we will assume the following assumption:

For every $\vartheta \in \Theta$, the stochastic part of $M(\vartheta)$ (i.e. the part depending on the white noise $\left.e(t) \sim W N\left(0, \lambda^{2}\right)\right)$ is \textbf{canonical} and has \textbf{no zeroes on the unit circle}.

The requirement that there are no zeroes on the unit circle instead poses some limitations on the systems we can identify. However:
\begin{itemize}
	\item zeroes on the unit circle are not usually required to model the behavior of a given system
	\item the behavior of models with zeroes on the unit circle can be approximate by means of models with zeroes close to the unit circle
\end{itemize} 

\subsection{PEM (Prediction Error Minimization) identification}

Paradigm to map data into a value of $\overline{\vartheta}$.
$$
	D^N=\left\{\overline{y(1)},\ldots,\overline{y(N)},\overline{u(1)},\ldots,\overline{u(N)}\right\}
$$
$D^N$ is a finite sequence of real numbers, observation of $y$ and $u$ over some horizon. $N$ is the length of the dataset.

\fg{0.3}{Screenshot (27)}

How to compare $\overline{y(1)},\ldots,\overline{y(N)}$ with $y(1,s),\ldots,y(N,s)$?

\fg{0.7}{Screenshot (28)}

\textbf{Idea}(Prediction Error Minimization)

$M(\vartheta)\implies \hat{M}(\vartheta)$ from stochastic models to predictive models.

ARMAX in $\mathcal{M}$ for a given value of $\vartheta$:

\begin{align*}
	M(\vartheta)&:\quad y(t)=\frac{B(z, \vartheta)}{A(z, \vartheta)} u(t-d)+\frac{C(z, \vartheta)}{A(z, \vartheta)} e(t)\\
	&\qquad\Downarrow\\
	\hat{M}(\vartheta)&: \quad \hat{y}(t \mid t-1) =\frac{B(z,\vartheta) E(z,\vartheta)}{C(z,\vartheta)} u(t-d)+\frac{F(z,\vartheta)}{C(z,\vartheta)} y(t-1) \qquad \text{one step prediction}
\end{align*}

\fg{0.7}{Screenshot (29)}


=======
$$y(t)=\underbrace{\frac{B(z)}{A(z)} u(t+k-d)}_{\text{This part of the process 
		is deterministically 
		known, and hence can 
		be trivially predicted}} +z(t+k)$$
>>>>>>> Stashed changes

%!TEX root = ../main.tex

To sum up, \textit{\gls{pem} identification} is a common paradigm for finding suitable models from data, based on prediction theory.

We have to move from stochastic models (ARMAX models) of type: 

$$
\Mc(\theta):\quad
y(t) = \frac{B(z,\theta)}{A(z,\theta)} u(t-d) +\ \frac{C(z,\theta)}{A(z,\theta)} e(t) ,\ e(t)\sim \WN(0,\lambda ^{2})
$$

to models in prediction form, i.e. the optimal predictors obtained through the theory we developed. In particular, we can focus on the \textbf{one-step predictor} already introduced:

$$
\hat{\Mc}(\theta):\quad
\hat{y}(t\mid t-1) = \frac{B(z,\theta) E(z,\theta) \ }{C(z,\theta)} u(t-d) + \frac{F(z,\theta)}{C(z,\theta)} y(t-1)
$$

The predictor has a different structure: while $ \Mc(\theta)$ is fed by a \gls{wn} and an input $u$, the predictor $\hat{\Mc}(\theta)$ is fed by measurements of $u$ and $y$ and returns the predicted future values of output, \textbf{there isn't a \gls{wn}}. The predictor model is returning the predicted values for \textit{that} realization of input and output.

\begin{figure}[htpb]
	\centering
	\begin{subfigure}{.5\textwidth}
		\centering
		\begin{tikzpicture}
			% place nodes
			\node [block] (m) at (0,0) {$\mathcal{M}(\vartheta)$};
			% connect nodes
			\draw [stealth-] (m.west) -- ++(-2,0) node[midway,above] {$u(t)$};
			\draw [stealth-] (m.north) -- ++(0,1) node[midway,right] {$e(t)$};
			\draw [-stealth] (m.east) -- ++(2,0) node[midway,above] {$y(t)$};
		\end{tikzpicture}
	\end{subfigure}%
	\begin{subfigure}{.5\textwidth}
		\centering
		\begin{tikzpicture}
			% place nodes
			\node [block] (m) at (0,0) {$\hat{\mathcal{M}}(\vartheta)$};
			% connect nodes
			\draw [stealth-] ([shift={(0, 0.2)}]m.west) -- ++(-2,0) node[midway,above] {$u(t-d)$};
			\draw [stealth-] ([shift={(0,-0.2)}]m.west) -- ++(-2,0) node[midway,below] {$y(t-1)$};
			\draw [-stealth] (m.east) -- ++(2,0) node[midway,above] {$\hat{y}(t\mid t-1)$};
		\end{tikzpicture}
	\end{subfigure}
\end{figure}
\FloatBarrier

The idea is that, when dealing with a real system, the values of $u$ and $y$ from time $t=1,\ldots,n$ can be collected, subsequently taking the predictor (introducing proper delays $ z^{-1}$) and feeding it with the measurement of $u$ and $y$ collected during the experiment. Then, the output of the predictor $\hat{\Mc}(\theta $) can be used to construct the value of the \textbf{prediction error} $ \varepsilon $. 
\begin{figure}[htpb]
	\centering
	\begin{tikzpicture}

		\node (origin) at (0,0) {};
		\node[cloud, draw, right=1.5cm of origin,
		    minimum width = 3cm,
	        minimum height = 2cm] (c) {$S$};
		
		\draw[-stealth] (origin.east) -- (c.west)
		    node[midway,above] {$u(t)$}
		    node[midway,below] {$ \begin{array}{c} \bar{u}(1)\\ \vdots \\ \bar{u}(n) \end{array}$}
		    node[near end] (u) {}
		    node[at end] (inputR) {};
		    
		\node[sum,below right = 1cm and 6cm of c.east] (s) {};
		\node[draw,
		minimum width=2cm,
	    minimum height=1.5cm,below left = 1cm and 1cm of s.center] (M) {$\hat{\mathcal{M}}(\theta)$};
	    
		\node[block,above left = 0.1cm and 1cm of M.west] (z1) {$z^{-1}$};
		\node[block,below left = 0.1cm and 1cm of M.west] (z2) {$z^{-1}$};
		
		\draw[-stealth] (c.east)-|(s.north)
		    node[pos=0.05] (fed) {}
		    node[near start,above] {$y(t)$}
		    node[near start,below] {$ \begin{array}{c}\bar{y}(1)\\ \vdots \\ \bar{y}(n)\end{array}$}
		    node[very near end, right] {$+$}
		;
		
		\draw[-stealth] (fed.center) |- (z1.west);
		\draw[-stealth]   (u.center) |- (z2.west);
		
		\draw[-stealth] (z1.east) -- ++(1,0);
		\draw[-stealth] (z2.east) -- ++(1,0);
		
		\draw[-stealth] (M.east)-|(s.south)
		    node[midway,right] {$\hat{y}(t\mid t-1,\theta)$}
		    node[very near end, right] {$-$}
		;
		
		\draw[-stealth] (s.east) -- ++(3,0)
		    node[midway,above]{$\varepsilon (t\mid t-1,\theta)$}
		    node[very near end] (startMin) {}
		;

		\draw[dashed,-stealth]
		    (startMin) -- ++(0,-3) -- ++(-5,0) 
		    node[midway,above]{$\min$}
		    -- ++(0,0.5)
		;
	\end{tikzpicture}
	\caption{The \gls{pem} identification scheme.}
\end{figure}
% \FloatBarrier

\begin{center}

\begin{tabular}{ccccc}
\toprule 
 $t$ & $u$ & $y$ & $ \hat{y}$ & $ \varepsilon $ \\
\midrule 
 $1$ & $ u(1)$ & $ y(1)$ & $ \hat{y}(1\mid 0)$ & $ \varepsilon (1\mid 0,\theta)$ \\
$2$ & $ u(2)$ & $ y(2)$ & $ \hat{y}(2\mid 1)$ & $ \varepsilon (2\mid 1,\theta)$ \\
$ \vdots $ & $ \vdots $ & $ \vdots $ & $ \vdots $ & $ \vdots $ \\
$n$ & $ u(n)$ & $ y(n)$ & $ \hat{y}(n\mid n-1)$ & $ \varepsilon (n\mid n-1,\theta)$ \\
 \bottomrule
\end{tabular}
\end{center}

All the values $ \hat{y} ,\ \varepsilon $ depend on the chosen $ \theta $. By inspecting the value of $ \varepsilon $, $ \theta $ can be tuned to make the prediction error as small as possible through a minimization process.

\textbf{We want to choose $\hat{\theta}_{N}$ that minimizes the prediction error.}

We have to introduce a metric to quantify the error. We define a \textbf{cost function:} 
\begin{equation*}
	\boxed{J_{N}(\theta) =\frac{1}{N}\sum _{i=1}^{N}(y(i) -\hat{y}(i\mid i-1,\theta))^{2} =\frac{1}{N}\sum _{i=1}^{N} \varepsilon (i\mid i-1,\theta)^{2}}
\end{equation*}
also known as \textbf{\gls{pem} minimization cost}, which can be interpreted as the \emph{empirical variance} of the prediction error over the collected data. Then:
\[
	\boxed{\hat{\theta }_{N} =\underset{\theta \in \Theta}{\argmin} J_{N}(\theta) =\underset{\theta \in \Theta}{\argmin}\frac{1}{N}\sum _{i=1}^{N} \varepsilon (i\mid i-1,\theta)^{2}}
\]

\subsection{Estimation of \texorpdfstring{$\lambda$}{lambda}}

$ \hat{\Mc}(\theta)$ depends only on $ \theta $ and the minimization of $J_{N}(\theta)$ returns only the best estimate for $\theta$, the part of the model that we need in order to reconstruct the predictor.

However, if we're interested in the description of the complete ARMA process, also $\lambda^{2}$ has to be estimated. 
\begin{equation*}
	\boxed{\hat{\lambda }_{N}^{2} =J_{N}(\hat{\theta }_{N}) =\frac{1}{N}\sum _{i=1}^{N} \varepsilon (i\mid i-1,\hat{\theta }_{N})^{2}}
\end{equation*}
The proposed estimate is the \textbf{empirical variance} of the prediction error for the optimal model. The idea behind the formula is simple: suppose $ S\in \Mc $ (i.e. our model class is rich enough to describe perfectly the true mechanism by which $y$ is generated) and $\hat{\theta }_{N}$ is so good that $\Mc(\hat{\theta }_{N}) =S$ (i.e. the information collected is enough to unveil $S$).

Since $ S\in \Mc \Longrightarrow S$ is an ARMAX process. Thus, there is a white noise $ \xi (t) \sim \WN(0,\lambda ^{2})$ in the real world that generates $y$. Then $ \hat{y}(t\mid t-1,\hat{\theta }_{N})$ is the optimal predictor not only for the model, but also for the system $S$. Most importantly
\begin{equation*}
\varepsilon (t\mid t-1,\hat{\theta }_{N}) =\xi (t)
\end{equation*}
the one-step prediction error in optimal prediction \textit{is} the \gls{wn} in the system. It makes sense to approximate the true variance by means of an empirical variance:
\begin{gather*}
\lambda ^{2} =\mathbb{E}\left[ \xi (t)^{2}\right] =\mathbb{E}[ \varepsilon (t\mid t-1,\hat{\theta }_{N})^2]\\
\hat{\lambda }_{N}^{2} =J_{N}(\hat{\theta }_{N}) =\frac{1}{N}\sum _{i=1}^{N} \varepsilon (i\mid i-1,\hat{\theta }_{N})^{2}
\end{gather*}

\subsection{Least Squares Identification (AR/ARX processes)}
Now that we have our model, we need to study its computational aspect: how can the cost function be minimized with respect to $ \theta ?$

In general $ J_{N}(\theta)$ is a very complicated and non-convex function of $ \theta $, with local minima and often without analytical or explicit expression. However, the subset of AR/ARX models, with good descriptive capabilities, has a quadratic cost function which can be explicitly minimized.

In this case \gls{pem} Identification takes the name of \textbf{Least Squares Identification.}

Given a generic ARX model
\begin{gather*}
y(t) =\frac{B(z,\theta)}{A(z,\theta)} u(t-d) +\frac{1}{A(z,\theta)} e(t) \qquad e(t) \sim \WN(0,\lambda ^{2})\\
A(z,\theta) =1-a_{1} z^{-1} -\cdots -a_{m} z^{-m}\\
B(z,\theta) =b_{0} +b_{1} z^{-1} +\cdots +b_{p} z^{-p}
\end{gather*}
with the black-box assumptions that we made, theta is the vector of the coefficients
\begin{equation*}
\theta = [ a_{1},\ldots,a_{m},b_{0},b_{1},\ldots,b_{p}]\transpose
\end{equation*}
The recursive equations associated to the model are
\begin{gather*}
A(z,\theta) y(t) =B(t,\theta) u(t-d) +e(t)\\
\left(1-a_{1} z^{1} -\cdots -a_{m} z^{-m}\right) y(t) =\left(b_{0} +b_{1} z^{-1} +\cdots +b_{p} z^{-p}\right) u(t-d) +e(t)\\
\Downarrow \\
y(t) =a_{1} y(t-1) +\cdots +a_{m} y(t-m) +b_{0} u(t-d) +b_{1} u(t-d-1) +\cdots +b_{p} u(t-d-p) +e(t)
\end{gather*}
If we introduce the vector of the regression variables $ \varphi $
\begin{equation*}
\varphi (t) =\begin{bmatrix}
y(t-1)\\
\vdots\\
y(t-m)\\
u(t-d)\\
u(t-d-1)\\
\vdots\\
u(t-d-p)
\end{bmatrix}
\end{equation*}
we can rewrite the recursive equations
\begin{equation*}
	y(t) =\underbrace{\varphi (t)\transpose \theta }_{\substack{\text{predictable}\\\text{at }t-1}} +\underbrace{e(t)}_{\substack{\text{unpredictable}\\\text{at }t-1}}
\end{equation*}
It is now possible to compute the one-step predictor for ARX model easily: we just need to delete the unpredictable part.
\begin{align*}
	\hat{\Mc}(\theta) :\ \hat{y}(t\mid t-1) & =a_{1} y(t-1) +\cdots +a_{m} y(t-m) +b_{0} u(t-d) +\cdots +b_{p} u(t-d-p)\\
	& =\varphi (t)\transpose \theta 
\end{align*}
The predictor is a very simple expression, a linear function of $ \theta $.

The identification cost is quadratic and positive:		
\begin{align*}
J_{N}(\theta) & =\frac{1}{N}\sum _{t=1}^{N}(y(t) -\hat{y}(t\mid t-1,\theta))^{2} & \\
 & =\frac{1}{N}\sum _{t=1}^{N}\left(y(t) -\theta \transpose \varphi (t)\right)^{2} & (\text{quadratic function of} \ \theta) \\
 & \geq \ 0 & \text{(sum of squares)}
\end{align*}


$ \hat{\theta }_{N}$ is the minimum of a quadratic and positive function, and is therefore determined by the first order condition:
\begin{equation*}
\frac{d}{d\theta } J_{N}(\theta) =\begin{bmatrix}
\frac{\partial J_{N}}{\partial a_{1}}(\theta)\\
\vdots \\
\frac{\partial J_{N}}{\partial b_{p}}(\theta)
\end{bmatrix} =0
\end{equation*}
i.e. a system of linear equations whose solutions are all and only minimizers of $ J_{N}(\theta)$.

By substitution:
\begin{align*}
\frac{d}{d\theta } J_{N}(\theta) & =\frac{d}{d\theta }[\frac{1}{N}\sum _{t=1}^{N}\left(y(t) -\theta \transpose \varphi (t)\right)^{2}] & \text{(linearity)}\\
 & =\frac{1}{N}\sum _{t=1}^{N}\frac{d}{d\theta }\left(y(t) -\theta \transpose \varphi (t)\right)^{2} & \\
 & =\frac{1}{N}\sum _{t=1}^{N} 2\left(y(t) -\theta \transpose \varphi (t)\right)\frac{d}{d\theta }\left(y(t) -\theta \transpose \varphi (t)\right) & \frac{d}{d\theta } y(t) =\begin{bmatrix}
0\\
\vdots \\
0
\end{bmatrix}\\
 & =\frac{1}{N}\sum _{t=1}^{N} 2\left(y(t) -\theta \transpose \varphi (t)\right)\frac{d}{d\theta }\left(-\theta \transpose \varphi (t)\right) & \frac{d}{d\theta }\left(-\theta \transpose \varphi (t)\right) =-\varphi (t)\\
 & =\frac{1}{N}\sum _{t=1}^{N} 2\underbrace{\left(y(t) -\varphi (t)\transpose \theta \right)}_{\text{scalar}}\underbrace{(-\varphi (t))}_{\text{column vec}} & \\
 & =\frac{2}{N}\sum _{t=1}^{N} \varphi (t)\left(\varphi (t)\transpose \theta -y(t)\right) & \\
 & =\frac{2}{N}\sum _{t=1}^{N} \varphi (t) \varphi (t)\transpose \theta \ -\frac{2}{N}\sum _{t=1}^{N} \varphi (t) y(t) & 
\end{align*} \ 
Setting $ \frac{d}{d\theta } J_{N}(\theta) =0$	


\begin{align*}
\cancel{\frac{2}{N}}\sum _{t=1}^{N} \varphi (t) \varphi (t)\transpose \theta  & =\cancel{\frac{2}{N}}\sum _{t=1}^{N} \varphi (t) y(t) & \\
\left[\sum _{t=1}^{N} \varphi (t) \varphi (t)\transpose\right] \theta  & =\sum _{t=1}^{N} \varphi (t) y(t) & \text{system of linear eqs in} \ \theta 
\end{align*}
This system of linear equations is commonly referred to as \textbf{Least Squares Normal Equations}. If 
\begin{equation*}
\sum _{t=1}^{N} \varphi (t) \varphi (t)\transpose \ \text{is non-singular}
\end{equation*}
the solution is unique, and $ \hat{\theta }_{N}$ is uniquely determined
\begin{equation*}
\boxed{\hat{\theta }_{N} =\left[\sum _{t=1}^{N} \varphi (t) \varphi (t)\transpose\right]^{-1}\sum _{t=1}^{N} \varphi (t) y(t)}
\end{equation*}
If $ \sum \varphi (t) \varphi (t)\transpose$ is singular, there are infinite solutions: $ \hat{\theta }_{N}$ can be determined by introducing a tie-break rule (e.g. taking the solution with minimum norm). This may happen when many models have the same predictive capability, such as in a redundant model class or in case of uninformative data.

\textbf{Geometric interpretation:}
\begin{equation*}
J_{N}(\theta) =\frac{1}{N}\sum _{t=1}^{N}\left(y(t) -\theta \transpose \varphi (t)\right)^{2}
\end{equation*}
is a paraboloid. The Hessian matrix $ \frac{d^{2}}{d\theta ^{2}} J_{N}(\theta)$ completely characterizes the space of quadratic functions.
\begin{align*}
\frac{d}{d\theta } J_{N}(\theta) & =\frac{2}{N}\sum _{t=1}^{N} \varphi (t) \varphi (t)\transpose \theta \ -\frac{2}{N}\sum _{t=1}^{N} \varphi (t) y(t)\\
\frac{d^{2}}{d\theta ^{2}} J_{N}(\theta) & =\frac{2}{N}\underbrace{\sum _{t=1}^{N} \varphi (t) \varphi (t)\transpose}_{\text{information matrix}}
\end{align*}
The information matrix is positive semi-definite. Indeed, taking a generic vector $x$ and remembering that $x\transpose \varphi (t) =\varphi (t)\transpose x$,
\[
	x\transpose\frac{d^{2}}{d\theta ^{2}} J_{N}(\theta) x = \frac{2}{N} x\transpose\left[\sum _{t=1}^{N} \varphi (t) \varphi (t)\transpose\right] x=\frac{2}{N}\sum _{t=1}^{N}\left(x\transpose \varphi (t)\right)^{2} \geq 0 \quad \forall x\neq 0
\]
$ J_{N}(\theta)$ is a paraboloid with minima:
\begin{figure}[htpb]
	\centering
	\begin{subfigure}{.5\textwidth}
		\centering
		\includegraphics[width=.8\linewidth]{paraboloid-proper}
		\captionof{figure}{Proper paraboloid}
		\label{fig:test1}
	\end{subfigure}%
	\begin{subfigure}{.5\textwidth}
		\centering
		\includegraphics[width=.8\linewidth]{paraboloid-degenerate}
		\captionof{figure}{Degenerate paraboloid}
		\label{fig:test2}
	\end{subfigure}
\end{figure}
\FloatBarrier
\begin{align*}
\frac{d^{2}}{d\theta ^{2}} J_{N}(\theta) & =\frac{2}{N}\sum _{t=1}^{N} \varphi (t) \varphi (t)\transpose \text{ non-singular (pos. def.)} & \Longrightarrow\quad& \text{proper paraboloid (unique min.)}\\
\frac{d^{2}}{d\theta ^{2}} J_{N}(\theta) & =\frac{2}{N}\sum _{t=1}^{N} \varphi (t) \varphi (t)\transpose \text{ singular} &\Longrightarrow\quad& \text{degenerate paraboloid}
\end{align*}
\input{lectures/2022_03_24}
\input{lectures/2022_03_28}
%!TEX root = ../main.tex

\section{Identification of ARMA/ARMAX models (Maximum Likelihood (ML) method)}

We consider a generic ARMAX:
\[
	M(\theta ): \quad y(t)=\frac{B(z,\theta )}{A(z,\theta)} u(t-d)+\frac{C(z,\theta )}{A(z,\theta)}e(t)\quad e(t)\sim \WN(0,\lambda^2)
\]
where
\begin{align*}
	A(z)&=1-a_{1} z^{-1}-a_{2} z^{-2}-\cdots-a_{m} z^{-m} \\
	B(z)&=b_{0}+b_{1} z^{-1}+b_{2} z^{-2}+\cdots+b_{p} z^{-p} \\
	C(z)&=1+c_{1} z^{-1}+c_{2} z^{-2}+\cdots+c_{n} z^{-n}
\end{align*}
We assume that $C(z,\theta)\neq 1$. Since $A$ and $C$ are monic the first term of the long division always gives $1$ and the remainder is always $C-A$. The predictor is then:
\[
	\hat{M}(\theta): \quad \hat{y}(t \mid t-1, \theta)=\frac{C(z,\theta)-A(z,\theta)}{C(z,\theta)} y(t)+\frac{B(z,\theta)}{C(z,\theta)} u(t-d)
\]
And the prediction error is:
\begin{align*}
	\varepsilon(t, \theta)=y(t)-\hat{y}(t \mid t-1, \theta)&=\left[1-\frac{C(z,\theta)-A(z,\theta)}{C(z,\theta)}\right] y(t)-\frac{B(z,\theta)}{C(z,\theta)} u(t-d)\\
	&=\frac{A(z,\theta)}{C(z,\theta)} y(t)-\frac{B(z,\theta)}{C(z,\theta)} u(t-d)
\end{align*}
Since we have $C(z,\theta)$ in the denominator, the error is \emph{no more linear} in $\theta$.
The cost function is \emph{non-convex} and may present \emph{local} minima:
\[
	J_{N}(\theta)=\frac{1}{N} \sum_{t=1}^{N} \varepsilon(t, \theta)^{2}
\]
To tackle the non-linearity we can use some kinds of numerical optimization using descent methods:
\begin{itemize}
	\item the algorithm is initialized with an initial estimate (typically randomly chosen) of the optimal parameter vector: $\theta^{0}$;
	\item update rule: $\theta^{i+1} = f (\theta^{i})$;
	\item the sequence of estimates should converge to $\hat\theta_{N}$.
\end{itemize}
If there are local minima, we can just use an \emph{empirical} approach by running the algorithm with many different starting values getting a range of candidates of minimum values, hoping to explore the entire domain and not miss the global minimum. This of course comes with the cost of computational complexity.


\section{Asymptotic Analysis of PEM Identification}

Is $M(\hat\theta_{N})$ a good model for the process $y(t)$?
We can give an asymptotic answer as $N\to \infty$

\textbf{Assumption on the data generating system:}

$y(t),u(t)$ are \gls{ssp} generated by a \textbf{linear system:}\footnote{Not necessarily of the same type as in $M$.}
\[
	S:
	\begin{cases}
		y(t) = G(z)u(t) + H(z)e(t)\\
		u(t) = F(z)r(t) + S(z)e(t)
	\end{cases}
	\qquad
	\begin{array}{l}
		e(t)\sim \WN(0,\lambda^2)\\
		r(t)\sim \WN(0,\sigma^2)
	\end{array}
\]
where $G(z),H(z),F(z),S(z)$ are \textbf{asymptotically stable, rational} transfer functions.

The most typical cases are when:
\begin{itemize}
	\item $S(z)=0$ (\textbf{open-loop experiment})
	\fg{0.6}{Screen Shot 2022-04-02 at 10.44.03}
	\item $S(z)\neq 0$ (\textbf{closed-loop experiment}), $S(z)$ accounts for $u(t)$ depending on $e(t)$ because of the feedback.
	\fg{0.6}{Screen Shot 2022-04-02 at 10.45.22}
\end{itemize}
The measured data sequence corresponds to a particular \textbf{realization} of input/output signals of $S$:
\[
	D^{N}=
	\begin{cases}
	 	u(1),u(2),\ldots,u(N)\\
	 	y(1),y(2),\ldots,y(N)
	\end{cases}
	\quad
	\text{to be thought as}
	\quad
	\begin{cases}
	 	u(1,\overline{s}),u(2,\overline{s}),\ldots,u(N,\overline{s})\\
	 	y(1,\overline{s}),y(2,\overline{s}),\ldots,y(N,\overline{s})
	\end{cases}
\]
Hence also the predictor computed from the given realization should be thought as:
\[
	\hat{y}(i\mid i-1,\theta) = f(D^{N}) = \hat{y}(i\mid i-1,\theta,\overline{s})
\]
and also:
\begin{gather*}
	\varepsilon(i,\theta) = y(i,\overline{s}) - \hat{y}(i\mid i-1,\theta,\overline{s}) = \varepsilon(i,\theta,\overline{s})
	\quad
	J_{N}(\theta) = \frac{1}{N}\sum_{i=1}^{N} \varepsilon(i,\theta,\overline{s})^2 = J_{N}(\theta,\overline{s})\\
	\hat{\theta}_{N} = \argmin_{\theta} J_{N}(\theta,\overline{s}) = \hat{\theta}_{N}(\overline{s})
\end{gather*}
\fg{0.6}{Screen Shot 2022-04-02 at 11.01.06}
Depending on the experiment, different costs and minimizers. As $N\to \infty$, the sequence of minimizers \textbf{converges:}\footnote{In practice, a good number could be $N\geq 300$.}
\fg{0.6}{Screen Shot 2022-04-02 at 11.04.00}
Indeed we have the following result.
\begin{theorem}
	Under the current assumptions, as the number of data points becomes larger and larger, we have with probability one that:
	\[
		J_{N}(\theta,s) = \frac{1}{N}\sum_{i=1}^{N} \varepsilon(i,\theta,s)^2 \xrightarrow{N\to\infty} \E[\varepsilon(t,\theta,s)^2] = \overline{J}(\theta)
	\]
	The convergence is almost sure w.r.t. $s$, uniform w.r.t. $\theta$ (the error goes to $0$ with the same rate for all $\theta$).\\
	Moreover if we define the \textbf{set of all minimizers:}
	\[
		\Delta = \left\{ \theta ^{\star} : \overline{J}(\theta ^{\star})\leq \overline{J}(\theta),\forall \theta  \right\}
	\]
	we have:
	\[
		\text{distance}[\hat{\theta}_{N}(s), \Delta]\xrightarrow{N\to\infty} 0
	\]
\end{theorem}
As a corollary we have that if $\Delta ={\theta ^{\star} }$ (i.e. $J_{N}(\theta)$ has a unique minimum point), then:
\[
	\hat{\theta}_{N}(s) \xrightarrow{N\to\infty} \theta ^{\star} 
\]
almost surely and regardless of the experiment.

To sum up, as $N$ is large enough we can approximate $M(\hat{\theta}(s)\approx M(\theta ^{\star}$.

Question: is $M(\theta^{\star}$ satisfactory? Let's assume that the systeam we are dealing with belongs to the class in which we construct the model, $S\in \mathcal{M}$. This is an ideal situation, we have enough degrees of freedom to describe the mechanism by which $y(t)$ is generated. That means that $\exists\theta^{0}: M(\theta^{0}) = S$.
%!TEX root = ../main.tex
We want to use this theory to justify more formally the method we introduced.

Let's start under the hypothesis that our model class is rich enough to describe our system.
\begin{equation*}
S\in \Mc \quad \left(\text{i.e. }\exists \theta ^{0} \in \Theta :S\equiv y(t) =\frac{B\left(z,\theta ^{0}\right)}{A\left(z,\theta ^{0}\right)} u(t-d) +\frac{C\left(z,\theta ^{0}\right)}{A\left(z,\theta ^{0}\right)} e(t) ,e(t) \sim \WN \left(0,\lambda ^{2}\right)\right)
\end{equation*}
Our model is not just a model, but the true system through which $ y(t)$ is generated.

The expression of the asymptotic cost function is:
\begin{equation*}
\begin{aligned}
\bar{J}(\theta) & =\E\left[ \varepsilon (t,\theta)^{2}\right] =\E\left[(y(t) -\hat{y}(t\mid t-1,\theta))^{2}\right] & \\
 & =\E\left[\left(y(t) \pm \hat{y}\left(t\mid t-1,\theta ^{0}\right) -\hat{y}(t\mid t-1,\theta)\right)^{2}\right] & \\
 & =\E\left[\left(\textcolor[rgb]{1,0,0}{y(t) -\hat{y}\left(t\mid t-1,\theta^{0}\right)} +\hat{y}\left(t\mid t-1,\theta ^{0}\right) -\hat{y}(t\mid t-1,\theta)\right)^{2}\right] &  \begin{array}{l}
\textcolor[rgb]{1,0,0}{(*)}\text{:generated by } S=\Mc\left(\theta ^{0}\right)\\
\text{optimal 1-step pred. error is the } \WN
\end{array}\\
 & =\E\left[\left(e^{0}(t) +\hat{y}\left(t\mid t-1,\theta ^{0}\right) -\hat{y}(t\mid t-1,\theta)\right)^{2}\right] &  \begin{array}{l}
\hat{y} \text{ linear functions of } y(t-1),y(t-2) \ldots\\
\text{uncorrelated with } e^{0}(t)
\end{array}\\
 & =\E\left[ e^{0}(t)^{2}\right] +\E\left[\left(\hat{y}\left(t\mid t-1,\theta ^{0}\right) -\hat{y}(t\mid t-1,\theta)\right)^{2}\right] & \\
 & =\underbrace{\lambda {^{0}}^{2}}_{\text{constant}} +\underbrace{\E\left[\left(\hat{y}\left(t\mid t-1,\theta ^{0}\right) -\hat{y}(t\mid t-1,\theta)\right)^{2}\right]}_{\geq 0,\ =0\text{ if } \theta =\theta ^{0}} & 
\end{aligned}
\end{equation*}

So that
\begin{equation*}
\bar{J}\left(\theta ^{0}\right) = \lambda {^{0}}^{2} \leq \bar{J}(\theta)\  \forall \theta \quad\Longrightarrow\quad \theta ^{0} \text{ is a minimum of } \bar{J}(\theta)
\end{equation*}
Using \gls{pem} asymptotic theory
\begin{equation*}
\hat{\theta }_{N}\xrightarrow{N\rightarrow \infty} \Delta ,\quad\theta ^{0} \in \Delta 
\end{equation*}
If $ \Delta =\left\{\theta ^{0}\right\}$ is a singleton (i.e. $ \bar{J}$ has unique minimum) then:
\begin{align*}
	\hat{\theta }_{N} & \xrightarrow{N\rightarrow \infty} \theta ^{0}\\
	\Mc(\hat{\theta }_{N}) & \xrightarrow{N\rightarrow \infty} S
\end{align*}
Thus, the \gls{pem} identification is \textbf{asymptotically consistent}: if $ \Mc$ is rich enough and the number of collected data is big enough, \gls{pem} identification provides a good description of $ S$.



\section{Model order selection}

The identification of model consists of 4 steps:
\begin{enumerate}
\item data collection;
\item choice of $ \Mc$;
\item optimal model selection\footnote{\gls{pem}: $ \hat{\theta }_{N} =\underset{\theta \in \Theta }{\argmin} J_{N}(\theta)$};
\item model validation\footnote{evaluate the performance of $\Mc(\hat{\theta }_{N})$}.
\end{enumerate}

The process is not linear: it is common to loop from model validation to another choice of $ \Mc$ (if the model is not satisfactory) or even to data collection.

An important choice for $ \Mc$ is the model order selection. Given an ARMAX model (black-box case):
\begin{gather*}
\begin{aligned}
y(t)  & =\frac{B(z,\theta)}{A(z,\theta)} u(t-d) +\frac{1}{A(z,\theta)} e(t) & e(t) \sim \WN \left(0,\lambda ^{2}\right)\\
A(z,\theta)  & =1-a_{1} z^{-1} -\cdots -a_{m} z^{-m} & \\
B(z,\theta) & =b_{0} +b_{1} z^{-1} +\cdots +b_{p} z^{-p} & \\
C(z,\theta) & =1+c_{1} z^{-1} +\cdots +c_{n} z^{-n} & 
\end{aligned}\\
\end{gather*}
The model order is: $ (m,p,n)$, the delay $ d$ between $ I/O$ is neglected as it can be easily retrieved from data.

Notice that some trivial choices for model selection change the structure of the model:
\begin{equation*}
\begin{cases}
n=0 & \text{ARX}\\
n=0,p=0 & \text{AR}\\
n\neq 0,p=0 & \text{ARMA}
\end{cases}
\end{equation*}
However, we will deal with the case where the structure is fixed ($ n,m,p\neq 0\Longrightarrow$ ARMAX) and we just want to select a proper number. To keep the notation simple we will focus on one hyper-parameter ($ m=n=p$) without loss of generalization.

A naive idea is to use $ J_{N}$ as an indicator for model quality, selecting a maximum value for the model order. Cycling for $ m=1:\overline{m}$, where $\overline{m}$ is the maximum model order we allow for:

$ \mathtt{For }\ m=1:\overline{m}$
\begin{equation*}
\begin{aligned}
\Mc^{m} & =ARMAX(m) =\left\{\Mc(\theta) ,\theta \in \Theta ^{m}\right\}\\
\hat{\theta }_{N}^{m} & =\underset{\theta \in \Theta ^{m}}{\argmin}\frac{1}{N}\sum _{i=1}^{N}(y(i) -\hat{y}(i\mid i-1,\theta))^{2}
\end{aligned}
\end{equation*}
$ \mathtt{END}$

At the end of the cycle we select:
\begin{equation*}
\hat{\theta }_{N} =\hat{\theta }_{N}^{m} \text{ with lowest } J_{N}\left(\hat{\theta }_{N}^{m}\right)
\end{equation*}
\begin{obs}
	This naive idea does NOT work in practice: $ J_{N}\left(\hat{\theta }_{N}^{m}\right)$ is always decreasing with $ m$, so that the highest possible order is selected. As $m$ increases, the model runs in the problem of over-fitting: it's perfectly good for describing the data collected but it does not generalizes to new data. The model has overall a bad predictive performance on new data.
\end{obs}
\begin{figure}[htpb]
    \centering
    \includegraphics[width=0.5\linewidth]{Screenshot (33)}
    \caption{Example of function interpolation.}
\end{figure}
\FloatBarrier
A very simple but powerful idea to solve the problem is to perform \textbf{cross validation}.

Cross validation consists in evaluating the performance of an identified model on new data: the dataset is divided in two parts, using the first (\textbf{training dataset}) for selecting the parameters and the second (\textbf{validation dataset}) for cross validation.

The for-loop becomes:

$ \mathtt{For } \ m=1:\overline{m}$
\begin{gather*}
\Mc^{m} =\text{ARMAX}(m) =\left\{\Mc(\theta) ,\theta \in \Theta ^{m}\right\}\\
\hat{\theta }_{T}^{m} =\underset{\theta \in \Theta ^{m}}{\argmin} J_{T}(\theta) =\underset{\theta \in \Theta ^{m}}{\argmin}\frac{1}{M}\sum _{i=1}^{M}(y(i) -\hat{y}(i\mid i-1,\theta))^{2}\\
\text{evaluate } J_{V}\left(\hat{\theta }_{T}^{m}\right) =\frac{1}{N-M}\sum _{i=M+1}^{N}\left(y(i) -\hat{y}\left(i\mid i-1,\hat{\theta }_{T}^{m}\right)\right)^{2}
\end{gather*}
$ \mathtt{END}$

Where $ J_{V}$ is the performance of the model obtained from the training set evaluated on the validation dataset. Then we select
\begin{equation*}
m^{\star} =\underset{m}{\argmin} J_{V}\left(\hat{\theta }_{T}^{m}\right) ,\hat{\theta }_{N} =\hat{\theta }_{T}^{m^{\star}}
\end{equation*}

\begin{figure}[htpb]
    \centering
    \includegraphics[width=.8\linewidth]{Screenshot(32)}
\end{figure}
\FloatBarrier
The drawback of cross-validation is that the dataset has to be split: less information is used to tune the model and data collection can be expensive.


\subsection{Alternative approach: direct model order penalization}
The structure is similar, we run a cycle up to a maximum model order

$ \mathtt{For } \ m=1:\overline{m}$
\begin{equation*}
\begin{aligned}
\Mc^{m} & =\text{ARMAX}(m) =\left\{\Mc(\theta) ,\theta \in \Theta ^{m}\right\} & \\
\hat{\theta }_{N}^{m}  & =\underset{\theta }{\argmin}\frac{1}{N}\sum _{i=1}^{N}(y(i) -\hat{y}(i\mid i-1,\theta))^{2} & \text{all data are used to tune the model}\\
\text{evaluate } V(m) & =V(m,J_{N}(\hat{\theta }_{N}^{m})),\forall m & \text{model cost with model order penalization}
\end{aligned}
\end{equation*}
$ \mathtt{END}$

Then we select
\begin{equation*}
m^{\star} =\underset{m}{\argmin} V(m) ,\hat{\theta }_{N} =\hat{\theta }_{T}^{m^{\star}}
\end{equation*}
Some common choice are
\begin{align*}
\text{FPE} & =\text{Final Prediction Error} & \text{(probability)}\\
\text{AIC} & =\text{Akaike's Identification Criterion} & \text{(statistics)}\\
\text{MDL} & =\text{Minimum Description Length} & \text{(information theory)}
\end{align*}
\begin{equation*}
	\begin{array}{rcc}
		\text{FPE}(m) = & \frac{N+m}{N-m}          &  \cdot J_{N}\left(\hat{\theta }_{N}^{m}\right)\\
		\text{AIC}(m) = & 2\frac{m}{N}             &  +\ln\left(J_{N}(\hat{\theta }_{N}^{m})\right)\\
		\text{MDL}(m) = & \ln(N)\frac{m}{N}        &  +\ln\left(J_{N}(\hat{\theta }_{N}^{m})\right)\\
		                & \text{increasing with }m &  \text{decreasing with }m
	\end{array}
\end{equation*}
In fact, FPE and AIC are approximately equivalent:
\begin{align*}
\ln(\text{FPE}) & =\ln\left(\frac{N+m}{N-m} J_{N}(\hat{\theta }_{N}^{m})\right) \\
 & =\ln\left(\frac{1+\frac{m}{N}}{1-\frac{m}{N}} J_{N}(\hat{\theta }_{N}^{m})\right) \\
 & =\ln\left(1+\frac{m}{N}\right) -\ln\left(1-\frac{m}{N}\right) +\ln\left(J_{N}(\hat{\theta }_{N}^{m})\right) \quad m\ll N, \frac{m}{N} \approx 0\\
 & \approx \frac{m}{N} -\left(-\frac{m}{N}\right) +\ln\left(J_{N}(\hat{\theta }_{N}^{m})\right) \\
 & =2\frac{m}{N} +\ln\left(J_{N}(\hat{\theta }_{N}^{m})\right) = \text{AIC} & 
\end{align*}
MDL is equal to AIC with 2 replaced by $ \ln(N)$. Since typically $ \ln(N)  >2$ (i.e. dataset has more than 8 data points) MDL penalizes model order more than AIC and FPE.
\input{lectures/2022_03_31}
\input{lectures/2022_04_04}
\input{lectures/2022_04_05}
\input{lectures/2022_04_06}
\input{lectures/2022_04_07}
\input{lectures/2022_04_11}
\input{lectures/2022_04_12}
\input{lectures/2022_04_13}
\input{lectures/2022_04_14}
\input{lectures/2022_04_18}
\input{lectures/2022_04_19}
\input{lectures/2022_04_20}
\input{lectures/2022_04_21}
\input{lectures/2022_04_25}
\input{lectures/2022_04_26}
\input{lectures/2022_04_27}
\input{lectures/2022_04_28}
\input{lectures/2022_05_02}
\input{lectures/2022_05_03}
\input{lectures/2022_05_04}
\input{lectures/2022_05_05}
\input{lectures/2022_05_09}
\input{lectures/2022_05_10}
\input{lectures/2022_05_11}
\input{lectures/2022_05_12}
\input{lectures/2022_05_16}
\input{lectures/2022_05_17}
\input{lectures/2022_05_18}
\input{lectures/2022_05_19}
\input{lectures/2022_05_23}
\input{lectures/2022_05_24}
\input{lectures/2022_05_25}
\input{lectures/2022_05_26}
\input{lectures/2022_05_30}
\input{lectures/2022_05_31}
\input{lectures/2022_06_01}
\input{lectures/2022_06_02}
\input{lectures/2022_06_06}

\end{document}